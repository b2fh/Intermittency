\documentclass[11pt,a4paper]{article}
\pdfoutput=1
\usepackage{jcappub}

\usepackage{amsmath,amssymb}
\usepackage{graphicx}

\graphicspath{{tmp_figs/}}

\newcommand{\figref}[1]{figure~\ref{#1}}
%\newcommand{\eqref}[1]{eq.~\ref{#1}}

\newcommand{\qc}{\ensuremath{{\bf q}}}

\begin{document}

\title{NonGaussianity From Chaotic Ballistic Trajectories}
\abstract{BLAH BLAH BLAH BLAH BLAH BLAH BLAH BLAH}

\maketitle

\section{Outline}
\begin{itemize}
\item Introduction - Inflation is awesome.  Standard separate universe picture during inflation.  We extend this into the post-inflation regime.  Phenomena resulting from averaged billiard trajectories.  Natural mechanism to produce spatially intermittent signatures (from a subset of the initial packet of trajectories crossing a feature in the potential, from caustic formation, ect)
\item There's a nice story 
\end{itemize}

\section{Introduction}
\begin{itemize}
\item Basic idea is the splitting of the dynamics into long- and short-wavelengths (the so called long-short or fast-slow split)
\item Previously the short dynamics has been treated by running full lattice simulations with the long-wavelength modes modelled as a constant
\item The long-dynamics is then treated space-time point by spacetime point as ballistics whose dynamics is determined by the subhorizon (lattice) evolution
\item Since the subhorizon dynamics can be modulated by the
\item This is not conceptually clean in Fourier space!!!!
\item Our novelty is that the short dynamics can be treated as decoupled ballistic trajectories as well
\item This provides a novel conceptual framework from which to understand production of density perturbations from preheating
\item Also presents nice parallels with the collapse of cold dark matter (until the discrete nature of the dark matter particles becomes important)
\item Tremendous computational gains relative to the lattice approach.  For a lattice we must run a large number of points so that the sub-horizon PDF is correctly approximated by the lattice sites.  Here, we can run a much smaller number of trajectories and simply attach of probability weight to each of them to get the smoothed lattice dynamics.  
\end{itemize}


\section{Multiscale Hierarchies and Generation of Modulation Parameters}
Outline the basic scales in the problem and indicate that we have a multiple hierarchies to consider.  Also explain how we will basically view each spatial position as some individual ballistic trajectory.  This generalises the philosophy behind the separate universe approximation.  When going from the smaller length scales back up to larger length scales in the hierarchy, convolutions must be done.  This allows for entropy generation via coarse-graining.  The complicated motion of the trajectories then allows for the coarse-graining to in some sense operate very differently in different spatial locations.  This leads to $\zeta$ perturbations and a relationship to entropy production.  To recover the results for the CMB, we must then smooth the fine-grained distribution associated with $\bar{\chi}$ smoothed over a Hubble patch at the end of inflation, back up to a scale associated with the size of a CMB pixel.

\begin{figure}
  \includegraphics[width=0.9\linewidth]{{{multiscale_isocon}}}
  \caption{Multiscale structure of the isocurvature mode that acts as a modulator for the preheating dynamics.}
\end{figure}


\section{Chaotic Structures in Lattice Simulations and the Ballistic Approximation}

\begin{itemize}
\item Show nice figures (ie. the giga-figure) from full lattice simulations
\item Include the equivalent results if we instead use the ballistic approach to show that they agree.
\item In the full lattice description, we must track all of the individual lattice sites simultaneously as a very high-dimensional PDF (as an approximation to doing the PDFunctionals for full fields)
\item The ballistics allows us to reduce the study of this phase space into an extremely reduced phase space where we just need the one-point (joint) PDFs for the variables at each lattice site.  During the ballistic regime, this mapping is rigourous.  In non-ballistic regime will have leakage out of the reduced PDF due to couplings from derivative terms.
\item Evolving the individual trajectories is the Langevin approach (with no noise term after inflation).  Getting the PDFs is the Fokker-Planck approach.
\item Mention that this general framework also applies during inflation and provides a nice unification.
\item Transition in using rest of the paper to explore ballistic dynamics.
\end{itemize}

\begin{figure}
  \caption{Final comoving curvature perturbation as a function of modulating fields associated with the superhorizon isocurvature direction $\chi$ and coupling constant $g^2/\lambda$.}
\end{figure}

\begin{figure}
  \caption{Evolution of $a^2H$ (or $a^4\rho$) as a function of time from lattice simulations (left) and from ballistics (right).}
\end{figure}

\begin{figure}
  \caption{Effective nonGaussianity function after smoothing over an appropriate scale. (could put this in the section with actual maps)}
\end{figure}

\begin{figure}
  \caption{Distribution of mean contributions to the scalar field energy density as a function of mean scale factor, obtained from a full resolved numerical lattice simulation}
\end{figure}

\begin{figure}
  \caption{Evolution of gravitational ``momentum'' along a spike and non-spike trajectory.  {\bf For comparison we also include the results from an individual trajectory and an averaged trajectory bundle in a ballistic simulation.}}
\end{figure}


\section{Dynamics of Trajectories in Phase Space and Caustics}
Several regimes to consider
\begin{itemize}
\item Classical motion with stochastic noise term.  Occurs while inflating and subhorizon modes cross the horizon.
\item This stage generates the isocurvature mode that allows for the chaotic motion to be realised
\item Classical decoupled trajectories with no noise term.  After inflation ends, but before onset of strong inhomogeneities from preheating.
\item Post-inflation can extend the decoupled trajectories approximation into the subhorizon regime, in the limit that the zero mode is unstable (and continuity for k small, so those modes are also unstable)
\item Fully mode-mode coupled inhomogeneous evolution.
\end{itemize}

\subsection{General Description of Trajectory Dynamics, Singular Embeddings of Lagrangian Submanifolds, etc.}

\subsection{Conversion of Ballistic Trajectories into Three-Dimensional Dynamics}
{\bf This is basically the description in terms of PDFs}


\section{Curvature Perturbations from Ballistic Dynamics During Preheating}
Let's now consider how the trajectory dynamics and development of caustics manifests itself in preheating, and leads to the production of (non-Gaussian) curvature perturbations.
We focus on a specific model in this section, although from the discussion it should be evident that the generic features leading to the production of curvature perturbations will appear in a wide variety of models.
Ultimately, we want to compute the curvature perturbation $\zeta = \ln a|_{\rho}(\chi_i)$ subject to the constraint of some initial mean value
\begin{equation}
  \langle \zeta | \chi_i, \sigma_\chi \rangle \qquad \langle \delta\chi^2 \rangle = \sigma_\chi^2 \, .
\end{equation}
Here $\chi_i = \langle \chi \rangle$ is the mean value of $\chi$ on the end of inflation hypersurface and $\sigma$ represents the subhorizon variance in $\chi$.
{\bf In principle we should include the other fluctuations in here.  In particular, chidot fluctuations are non-negligible on subhorizon scales and these will also contribute.  The superhorizon chidot can be approximated with the slow-roll solution (although in practice the correlations can also be computed to do a more rigourous job).}
In order to emulate our lattice simulations, we will consider two hierarchies of smoothings.
In the first, we allow the mean field $\chi_i$ to vary at the start of the simulation.  At a fixed conformal time $\tau$ later we then extract whichever observable quantities we want by {\bf convolving with a Gaussian kernel on the initial trajectories in the cloud}.
\begin{equation}
  \langle\mathcal{O}|\chi_i,\sigma\rangle(\tau) = H_I^{-1}\int d^3x 
\end{equation}
These smoothed quantities then approximate the results that would be obtained from running a lattice simulation.
We are most interested in $\zeta(\chi_i)$, so we now evaluate these smoothed trajectory bundles at a fixed value of the smoothed density $\rho$.



    {\bf More generally, we should take an initial Gaussian cloud in all of the field phase space directions and follow this.  We don't need to follow $a^2H$ since that is constrained (and we just solve the constraint initially), and we could just take $\ln a$ as a time variable since it is monotonic, so we can safely set $a=1$ through a reparameterisation of time along the trajectory.  Alternatively, in the simulations $a$ is initially the same at each grid site.}

{\bf Emphasize the tremendous computational gain from this approach}

In the ballistic approximation, an $n$-field scalar field model interacting with gravity lives in a $4n+2$-dimensional phase space along with a single constraint $3H^2 = \rho$. {\bf This argument makes the manifold the motion actually occurs on odd dimensional.  This seems odd from a Hamiltonian point of view.  Does that mean the motion when the constraint is explicitly enforced lacks a symplectic structure?}.
To illustrate the basic phenomena, our single-particle Lagrangian to be (up to a total derivative)
\begin{equation}
  L = \frac{a^2}{2}\dot{\phi}^2 + \frac{a^2}{2}\dot{\chi}^2 - a^4\frac{\lambda}{4}\left(\phi^4 + \frac{g^2}{\lambda}\phi^2\chi^2\right) - 3M_P^2\dot{a}^2 + 3M_P^2\partial_\tau(a\dot{a})
\end{equation}
For simplicity, we set $M_P=1$ in what follows.  It is convenient to choose $\varphi_1 = a\phi$ and $\varphi_2 = a\chi$ as canonical field variables, with corresponding canonical momenta $\Pi_1 = a\dot{\phi}$ and $\Pi_2=a\dot{\chi}$ where a dot represents a derivative with respect to conformal time $\tau$.

In terms of these variables, the Hamiltonian is
\begin{equation}
  \mathcal{H} = \frac{\Pi_1^2}{2} + \frac{\Pi_2^2}{2} + \left(\varphi_1\Pi_1 + \varphi_2\Pi_2\right)\frac{\Pi_a}{a} + \frac{\lambda}{4}\left(\varphi_1^4 + \frac{2g^2}{\lambda}\varphi_1^2\vartheta_2^2 \right) - \frac{\Pi_a^2}{12 M_P^2}
\end{equation}
with the constraint
\begin{equation}
  \mathcal{H} = 0
\end{equation}
These yield the following equations of motion
\begin{align}\label{eqn:eom_scale}
  \frac{\partial\varphi_i}{\partial\tau} &= \Pi_i + \frac{\Pi_a}{a}\varphi_i \\
  \frac{\partial\Pi_i}{\partial\tau}     &= -\frac{\dot{a}}{a}\Pi_i - \frac{\partial \tilde{V}}{\partial\varphi} \\
  \frac{\partial a}{\partial\tau}        &= -\frac{\Pi_a}{6M_P^2} \\
  \frac{\partial\Pi_a}{\partial\tau}     &= \frac{\Pi_a}{a^2}\sum_i\varphi_i\Pi_i
\end{align}
where $\tilde{V}(\vec{\varphi}) = a^{-4}V(\vec{\phi})$.
Alternatively, in unscaled field variables $\phi,\chi$ we have
\begin{equation}
  \mathcal{H} = \sum_i\frac{\pi_i^2}{2a^2} + a^4V(\vec{\phi}) - \frac{\Pi_a^2}{12M_P^2}
\end{equation}
and equations
\begin{align}\label{eqn:eom_noscale}
  \frac{\partial\phi_i}{\partial\tau} &= \frac{\pi_i^2}{a^2} \\
  \frac{\partial\pi_i}{\partial_\tau} &= -a^4 \partial_iV(\phi) \\
  \frac{\partial a}{\partial\tau}     &= -\frac{\Pi_a}{6M_P^2} \\
  \frac{\partial\Pi_a}{\partial\tau}  &= \sum_i\frac{\pi_i^2}{a^3} - a^3V \, .
\end{align}
We solve the equations of motion~\eqref{eqn:eom_noscale} using a tenth order accurate Gauss-Legendre integrator~\cite{Butcher,Braden}.
This integration scheme is symplectic, preserving quadratic invariants of the action as well as the canonical symplectic 2-form for the Hamiltonian system.
As will be seen, this dynamical system exhibits chaotic behaviour for a range of coupling parameter $2g^2/\lambda$.
Therefore, the preservation of the phase-space structure by our integration scheme is of prime importance when attempting to compute averaged quantities over a cloud of initial trajectories,
since the chaotic motion would quickly destroy preservation of probabilities for an inferior integration scheme.
Due to the extreme accuracy of our integrator, we are able to preserve the Hamiltonian constraint $\mathcal{H}=0$ to machine precision along each individual ballistic trajectory.
{\bf Figure?  Can explicitly see the bit-flipping and corresponding random walk over time if I plot this.  Probably a good idea to also halve the time-step and compare trajectories.}

%In this sense, it is more natural to consider the equations for the longitudinal and transverse parts of the field {\bf Work out this and do the calculation}.
%An approximate, but less technically involved approach, is to instead consider the radial and angular parts of the field defined through
%\begin{align}
%  \phi &= \sigma\cos\theta \\
%  \chi &= \sigma\sin\theta
%\end{align}

\begin{figure}
  \caption{Caustic plot in $\Pi_a = a^2H$.  On the left show the ballistic trajectory one (with appropriate smoothings), on the right show the full lattice simulation one.}
\end{figure}
{\bf Explain the structure in this caustic plot.  The periodicity is easy to understand.  Obtaining the basic structural block requires numerics.  We should also be able to get the scaling of the peaks widths, etc through some reasonably simple argument.  The existence of two distinct peaks occurs because there are 2 arms? (could easily check with another model with 3 arms say).  Substructure comes from usual period doubling and extreme coiling of the string as time evolves.}

We now present the evolution for a single period of the caustic structure in the initial values of $\ln\chi_i$.
For our example, this period is controlled by the Floquet exponent $\mu T_\phi$ for linear perturbations in $\chi$ around a trajectory with $\chi=0$.
As is well known, this problem can be reduced to the solution of the Lame equation~\cite{KofmanGreene} up to small corrections associated with the evolution of the scale factor $a$.
From the ballistic viewpoint, this stage is more naturally viewed as a rotation of trajectory's oscillation ``plane'' {\bf line?} in both the $(\phi,\chi)$ and $(\dot{\phi},\dot{\chi})$ planes.

\begin{figure}
  \caption{Length of the phase string as a function of time.  Here we have defined our metric on phase space as $ds^2 = d\varphi_1^2+d\varphi_2^2 + \Pi_1^2 + \Pi_2^2$.  In addition to the full length of the string, we also show }
\end{figure}

\begin{figure}
  \caption{Response of the phase space variables to small initial perturbations in the modulating field value $\delta\chi$.}
\end{figure}

\begin{figure}
  \includegraphics[width=0.49\linewidth]{{{phase-string-3d_t600}}}
  \includegraphics[width=0.49\linewidth]{{{phase-string-3d_t625}}}
  \caption{Three-Dimensional Projection of the initial phase string into the $(a\phi,a\chi,a\dot{\phi})$ cube.  The full phase string is shown in black.  The projections onto three remaining two-dimensional planes are shown in blue.  Also included for reference are isocontours on constant potential energy, energy in $\phi$ (with $\chi=0$) and ..., along with the isocontour corresponding to the initial value of the energy.  Since the momenta associated with the scale factor $a$ is conserved in time average, these initial isocontours can be viewed as roughly showing the projection of the constraint surface $\mathcal{H}=0$ into the appropriate planes.  From left to right we show the distribution at $\tau = $,$\tau = $, and $\tau = $.}
\end{figure}

\begin{figure}
  \caption{Entropy of projected PDF as a function of time?}
\end{figure}

\begin{figure}
  \caption{Distribution of $\ln a$ for a Gaussian bundle of trajectories centered around a spike (\emph{left}) and without a spike (\emph{right}).}
\end{figure}

{\bf To Do:  Fix the initial Hubble to be the same at the start of each inidividual trajectory?  Might remove one of the ugly splitting issues.}

Things to understand
\begin{itemize}
\item What determines the width of the packet that enters into the arm?  This will somehow determine the strength and width of the spike.
\item The width of the packet in field space that goes in the arm can be related to the spatial size.  i.e. fraction of Hubble volume going into the arm.  This presumably tells us something about how the height of the spike scales and their width.
\end{itemize}

\subsection{Density Perturbations From Caustic Formation}
Stages of post-inflation dynamics.  These may or may not occur in a given model.
\begin{itemize}
\item Initially have Floquet theory.  At each passage through (minimum?) the length of the string increases, while maintaining roughly constant density.
\item During bouncing off hyperbolic walls, the field trajectories can split.  This leads to peak in the density of the string as a function of ``invariant'' distance.
\item Eventually get folding of trajectories.
\item It's important to have an instability in the zero mode, or there will be no splitting of trajectories.
\end{itemize}

The post-inflationary dynamics passes through a series of distinct stages.
Our focus here is on a specific model, but the general features we describe can be ubiquitous for a variety of potential minima.
The key feature for our analysis is the exponential divergence of neighbouring trajectories during the oscillations of the field around the potential minimum.
This requires an instability of the zero-mode of the 
In the initial stages of post-inflationary evolution, the field undergoes damped oscillations around the minimum of the potential.

For our specific model, the field initially oscillates primarily in the $\phi$ direction, and the zero-mode of the $\chi$ field (equivalently the angular direction for the small $\chi$ limit) experiences a parametric instability described by Floquet theory.
For certain choices of $g^2/\lambda$, the zero mode of $\chi$ is unstable, leading to a rotation of the line of oscillation for the individual trajectories at the bottom of the well.
The instability also causes the individual trajectories to diverge from each other, resulting in an elongation of the phase string.
{\bf This elongation is more or less uniform, and no sharp features are imprinted onto the density during this phase}.

The Floquet instability continues until the field trajectories have rotated enough to probe the hyperbolic parts of the potential isocurvature lines at the bottom of the potential.
Once this occurs, the linearised Floquet analysis breaks down and we enter the chaotic billiard dynamics regime.
The reflection of the individual lattice sites from the potential barriers acts as a softened version of the reflection of a billiard ball from the rails of a billiard table.
It is well known that billiard motion on a variety of non-rectangular tables (and in particular tables with hyperbolic wall segments) results in focussing and splitting of trajectories along the table.
In phase space, this corresponds to the folding and wrapping of the string, with the caustics resulting from the projection back to $\phi,\chi$ space corresponding to {\bf write this more eloquently}.

From our dynamical simulations, we can directly observe the build-up of these foldings and the resulting trajectory focussing and caustic formation.
This process is illustrated in~\figref{fig:}.

\begin{itemize}
\item Bounces off of walls now begin to split the trajectories at specific points, rather than uniformly along the entire string
\item Repeated bounces lead to folding of the string due to these splittings (ie. since it's inhomogeneous along the strong)
\item The folds project into caustics in the $\ln a$ plane
\item For $\bar{\chi}$ located near such a caustic value for $\chi_0$, there is a singular contribution to $\zeta$ when averaging over a horizon volume.  This occurs because many of the trajectories will have the same expansion history near the caustic (there is a bunching up in the potential).
\item For $\bar{\chi}$ away from such caustics, phase mixing in the bottom of the well smooths out the overall expansion history that that of a fluid with $w=1/3$ for the potential with dimensionless couplings that we use here.
\end{itemize}
{\bf Floquet phase of zero mode relies on the conformal rescaling.  However, the bifurcation dynamics does not.}

\subsection{Extension to Other Models}
The inflationary dynamics generating the density perturbations in the CMB can be decoupled from the potential shape during the post-inflation oscillations.
This will be generic if inflation occurs on a random potential such as in the landscape paradigm.
In particular, the inflaton may traverse a region with flat directions during inflation which are sufficient to generate superhorizon isocurvature modes which will act as a modulator for the preheating dynamics.

{\bf Need generation of the isocurvature modes, which act as modulators post inflation.}
{\bf Question: If I generate isocurvature modes then snap down to a trough, do the superhorizon fluctuations remain there?}

For example, if we restrict to two-field models we may consider the following parameterisation of the minimum
\begin{equation}
  V(\rho,\theta) = \sum_{m} R_m(\rho)\cos(m\theta + \alpha_m)
\end{equation}
with
\begin{equation}
  \phi_1 = \rho\cos\theta \qquad \phi_2 = \rho\sin\theta
\end{equation}
As an example, such potentials arise naturally if we consider Taylor expanding around the minimum
\begin{equation}
  V(\phi_{min}+\delta\phi) = V(\phi_{min})
\end{equation}
with each additional derivative term in $V$ contributing an additional Chebyshev mode to our expansion.
However, the higher terms in the Taylor expansion also include contributions to the lower Chebyshev modes.
Due to the nice numerical properties of choosing non-monomial potentials as a basis, it is instead nicer to work with a Chebyshev expansion.

\subsection{Include Critical Lines Flowing Into the Minimum}

\begin{figure}
  \caption{Various phase space projections of the dynamics.}
\end{figure}

\begin{figure}
  \caption{Show ``length'' of the string in phase space (under some definition) as a function of time.  This will play nicely with the density.  The question is what to use for density measurements.  Perhaps it's useful to show the length projected into a number of different planes?}
\end{figure}

\begin{figure}
  \caption{Show splitting of trajectories according to which arm they enter.}
\end{figure}

\begin{figure}
  \caption{Poincare slices}
\end{figure}

\begin{figure}
  \caption{Nice picture of caustic formation.}
\end{figure}

\begin{figure}
  \caption{Show trajectory of bifurcation points (ie caustics on initial string) in the potential}
\end{figure}

\begin{figure}
  \caption{Show the effective potential.}
\end{figure}

\section{Density Perturbations from Features in the Inflationary Potential}
We have presented a generic mechanism for generating adiabatic density perturbations from the nonlinear dynamics of oscillating scalar fields after inflation.
The ballistic approach used here relies on the presence of instabilities in the zero-mode dynamics of the field,
a mechanism to generate initial field fluctuations that are able to act as a modulator for the posti inflation dynamics,
and {\bf way to convert chaotic behaviour and caustics into actual differences in expansion history}.
However, similar considerations naturally apply to inflationary dynamics themselves.
In this case, the well-known picture of stochastic inflation {\bf yadda yadda, blah blah}.
{\bf Expand on how to apply this viewpoint to stochastic inflation as well.}

\section{CMB Maps from Generalised Local NonGaussianity}
\begin{itemize}
\item Description of the form of nonGaussianity produced by these models.
\item Description of how to do marginalisation from coarse-grained distribution (ie. smoothed over a Hubble at the end of inflation) to pixels in CMB of LSS
\item T and E maps from generalised local nonGaussianity
\end{itemize}


\section{Curvature Perturbation, Entropy Generation, and }
The evolution of the large-scale comoving curvature perturbation can be related to the production of entropy in the coarse-grained system.
From the second law of thermodynamics, we have
\begin{equation}
  d\rho + (\rho+P)d\ln V = \frac{TdS}{V}
\end{equation}
and along individual ballistic trajectories, comoving conservation of stress-energy enforces
\begin{equation}\label{eqn:rho-cons}
  \dot{\rho} + 3H(\rho+P) = 0
\end{equation}
which demonstrates that the entropy is conserved along individual ballistic trajectories.
It is convenient to define
\begin{equation}
  d\zeta = \frac{d\ln\rho}{3(1+w)} + \frac{1}{3}d\ln V
\end{equation}
so that along individual paths we have
\begin{equation}
  \zeta(t) \equiv \int_{t_i}^t dt \frac{1}{3(1+w)}\frac{d\ln\rho}{dt} + \frac{d\ln a}{dt} \, .
\end{equation}
From~\eqref{eqn:rho-cons}, we see that $\zeta$ is conserved along ballistic trajectories as long as the the trajectories do not interact with each other and gradient terms can be neglected.
However, when gradients in the field become important, this trajectory wise conservation is broken.
In a simplistic case where we assume the metric continues to be FRW, but allow the fields $\phi_i$ to be inhomogeneous, we have
\begin{equation}
  \frac{1}{3(1+w)}\frac{d\ln\rho}{dt} + \frac{d\ln a}{dt} = \sum_i\frac{\nabla\cdot(\dot{\phi_i}\nabla\phi_i)}{3a^2(\rho+P)}
\end{equation}
for the case of a collection of scalar fields with canonical kinetic term.
Beyond the ballistic approximation, the individual trajectories will be connected to each other via spatial gradients (and non-locally through the Hubble constraint).

When considering the effects of divergences between small scale trajectories on the overall large-scale expansion, it is convenient to remove the effects of the current being exchanged between the trajectories by writing
\begin{equation}
  \frac{d\zeta}{dt} = \frac{1}{a^2}\nabla\cdot\left(\frac{\dot{\phi}_i\nabla\phi_i}{3(\rho+P)}\right) + \frac{1}{3a^2}\frac{\dot{\phi}\nabla\phi}{\rho+P}\cdot\nabla\ln(\rho+P) \, .
\end{equation}
We therefore define the $\zeta$ current
\begin{equation}
  {\bf J}_{\zeta} \equiv \frac{\dot{\phi}\nabla\phi}{3(\rho+P)}
\end{equation}
so that
\begin{equation}
  \frac{d\zeta}{dt} = \nabla\cdot{\bf J}_\zeta + {\bf J}_\zeta\cdot\nabla\ln(\rho+P) \, .
\end{equation}
If we consider $\zeta$ averaged on a constant $t$ slice, we then find
\begin{equation}
  \frac{d\langle\zeta\rangle_V}{dt} = \frac{1}{V}\int (1+\ln(\rho+P)){\bf J}_\zeta\cdot d{\bf \Sigma} - \frac{1}{V}\int d{\bf x}\ln(\rho+P)\nabla\cdot{\bf J}_\zeta \, .
\end{equation}
If we further define the $\zeta$ potential by
\begin{equation}
  \nabla^2\Psi_\zeta = \nabla\cdot{\bf J}_\zeta
\end{equation}
then we can alternatively write the volume term as
\begin{equation}
  \frac{1}{V}\int d{\bf x}\ln(\rho+P)\nabla\cdot{\bf J}_\zeta = \frac{1}{V}\int d{\bf x}\Psi_\zeta \nabla^2\ln(\rho+P) \, .
\end{equation}

{\bf Include more general case.  Allow for projections into fluid rest frame, etc., figure out which is the best frame to be using for this analysis}


\subsection{Some Interesting Musings}
First consider a large volume of space, denoted by $\Omega_{UL}$.
Define a probability distribution $Q_{UL}(x) = a^3(x) / V_{UL}$ where $V_{UL} = \int_{\Omega_{UL}}d^3x a^3(x)$ is the volume of some superhorizon region encompassing at least our current Hubble volume.
This choice corresponds to weighting individual lattice sites by their relative increase in volume from some initial surface where they all have uniform size.
Now consider a partition of $\Omega_{UL}$ which we denote $\Omega = \{\Omega_i\}$ with $\cup_i \Omega_i = \Omega$ and $\Omega_i \cap \Omega_j = \delta_{ij}\Omega_i$.
This should be viewed as a crude approximation to performing a spatial averaging by convolution with a window function (which we will return to shortly).
Define a coarse-grained probability distribution associated with the partition $\Omega$ by $Q_\Omega(x) = V_i / V_{UL}$ where $i$ is chosen by the condition $x \in \Omega_i$.
The Kullback-Leibler divergence of $Q_\Omega$ from $Q_{UL}$ is
\begin{equation}
  D(Q_{UL} \parallel Q_\Omega) = \int_{\Omega_{UL}} d^3x Q_{UL}\ln\left(\frac{Q_{UL}}{Q_\Omega}\right) = \sum_i \left(\frac{V_i}{V_{UL}}\right)\left[\frac{3}{V_i}\int_{\Omega_i}d^3xa^3\ln a -\ln V_i\right]
\end{equation}

We can also define normalised probabilities on each subset in the partition $Q_i = a^3 / V_i$.
\begin{equation}
  \frac{1}{V_i}\int_{\Omega_i} d^3x a^3\ln a = \frac{1}{3} \int d^3x Q_i\ln Q_i + \frac{1}{3}\ln V_i
\end{equation}
where we've defined
\begin{equation}
  V_i = \int_{\Omega_i} d^3x a^3 \, .
\end{equation}
Therefore, the perturbation to $\zeta$ relative to what it would have been for a completely homogeneous evolution (with each trajectory experiencing the same expansion history) is encoded by the Shannon entropy associated with weighting individual trajectories by their overall volume expansion.

However, we may also wish to compare $\ln a$ in $\Omega$ to the average expansion in the ultra-large volume instead.
\begin{equation}
  \delta\zeta \equiv \langle \ln a \rangle_\Omega - \langle \ln a \rangle_{V_{UL}}
\end{equation}

\begin{figure}
  \caption{Here we initialise a single sine wave of fixed wavenumber as the initial condition for our ballistics code.  We can then evolve it to show how initially it's just linear perturbation theory (and can even FT to get the growth of higher harmonics) and then compare this to a lattice simulation with the identical Fourier mode initialized.  This will help clarify exactly how the zero-mode is capturing the growth of inhomogeneities, i.e. if the restoring forces from gradients are actually important in determining the growth rates of various Fourier modes.  Probably the point is that nonlinearity in the background onsets well before nonlinearity from mode-mode coupling, and this is what drives everything.}
\end{figure}

\bibliography{ballistics}
\bibliographystyle{plain}

\end{document}
