\documentclass[11pt,a4paper]{article}
\pdfoutput=1
\usepackage{jcappub}

\usepackage{amsmath,amssymb}
\usepackage{graphicx}

\newcommand{\figref}[1]{figure~\ref{#1}}
%\newcommand{\eqref}[1]{eq.~\ref{#1}}

\newcommand{\qc}{\ensuremath{{\bf q}}}

\begin{document}

\title{NonGaussianity From Chaotic Ballistic Trajectories}
\abstract{BLAH BLAH BLAH BLAH BLAH BLAH BLAH BLAH}

\maketitle

\section{Outline}
\begin{itemize}
\item Introduction - Inflation is awesome.  Standard separate universe picture during inflation.  We extend this into the post-inflation regime.  Phenomena resulting from averaged billiard trajectories.  Natural mechanism to produce spatially intermittent signatures (from a subset of the initial packet of trajectories crossing a feature in the potential, from caustic formation, ect)
\item There's a nice story 
\end{itemize}

\section{Introduction}
BLAH BLAH BLAH BLAH BLAH BLAH BLAH BLAH

\section{Multiscale Hierarchies and Generation of Modulation Parameters}
Outline the basic scales in the problem and indicate that we have a multiple hierarchies to consider.  Also explain how we will basically view each spatial position as some individual ballistic trajectory.  This generalises the philosophy behind the separate universe approximation.  When going from the smaller length scales back up to larger length scales in the hierarchy, convolutions must be done.  This allows for entropy generation via coarse-graining.  The complicated motion of the trajectories then allows for the coarse-graining to in some sense operate very differently in different spatial locations.  This leads to $\zeta$ perturbations and a relationship to entropy production.  To recover the results for the CMB, we must then smooth the fine-grained distribution associated with $\bar{\chi}$ smoothed over a Hubble patch at the end of inflation, back up to a scale associated with the size of a CMB pixel.

\begin{figure}
  \includegraphics[width=0.9\linewidth]{{{multiscale_isocon}}}
  \caption{Multiscale structure of the isocurvature mode that acts as a modulator for the preheating dynamics.}
\end{figure}


\section{Chaotic Structures in Lattice Simulations and the Ballistic Approximation}

\begin{itemize}
\item Show nice figures (ie. the giga-figure) from full lattice simulations
\item Include the equivalent results if we instead use the ballistic approach to show that they agree.
\item In the full lattice description, we must track all of the individual lattice sites simultaneously as a very high-dimensional PDF (as an approximation to doing the PDFunctionals for full fields)
\item The ballistics allows us to reduce the study of this phase space into an extremely reduced phase space where we just need the one-point (joint) PDFs for the variables at each lattice site.  During the ballistic regime, this mapping is rigourous.  In non-ballistic regime will have leakage out of the reduced PDF due to couplings from derivative terms.
\item Evolving the individual trajectories is the Langevin approach (with no noise term after inflation).  Getting the PDFs is the Fokker-Planck approach.
\item Mention that this general framework also applies during inflation and provides a nice unification.
\item Transition in using rest of the paper to explore ballistic dynamics.
\end{itemize}

\begin{figure}
  \caption{Final comoving curvature perturbation as a function of modulating fields associated with the superhorizon isocurvature direction $\chi$ and coupling constant $g^2/\lambda$.}
\end{figure}

\begin{figure}
  \caption{Evolution of $a^2H$ (or $a^4\rho$) as a function of time from lattice simulations (left) and from ballistics (right).}
\end{figure}

\begin{figure}
  \caption{Effective nonGaussianity function after smoothing over an appropriate scale. (could put this in the section with actual maps)}
\end{figure}

\begin{figure}
  \caption{Distribution of mean contributions to the scalar field energy density as a function of mean scale factor, obtained from a full resolved numerical lattice simulation}
\end{figure}

\begin{figure}
  \caption{Evolution of gravitational ``momentum'' along a spike and non-spike trajectory.  {\bf For comparison we also include the results from an individual trajectory and an averaged trajectory bundle in a ballistic simulation.}}
\end{figure}


\section{Effective One-Particle Dynamics, Phase Space Trajectories and Singular Embeddings}
In the ballistic regime we treat the field values at each individual lattice site independently.
Ultimately, observations will be tied to quantities averaged over many post-inflationary Hubble volumes, and therefore do not require detailed knowledge of the fine-grained dynamics of the phase space trajectories at each point in space.
This allows us to extract all relevant quantities from the effective one-particle phase space, and we may consider the motion of bundles of trajectories rather than full lattice simulations.

Spatial averages are replaced by integrals over effective one-particle probability distributions
\begin{equation}
  \langle\mathcal{O}\rangle_V \equiv \frac{1}{V}\int_V \mathcal{O}(x) d^dx \to \int d\mathcal{O} \mathcal{P}_{\mathcal{O}}(\mathcal{O},t)\mathcal{O} = \int d\qc \mathcal{P}_\qc(\qc,t)\mathcal{O}(\qc) \, .
\end{equation}
The phase-space probability distribution $\mathcal{P}(\qc,t)$ at any time $t$ is related to the initial distribution $\mathcal{P}(\qc,t_0)$ by
\begin{equation} \label{eqn:probability-transport}
  \mathcal{P}(\qc_f,t_f) = \int d\qc_0\delta\left(\qc_f-\qc(t|\qc_0)\right)\mathcal{P}(\qc_0,t_0) \, .
\end{equation}
{\bf Initial distribution comes from statistics of initial field distribution.  Expand on this point}
The field dynamics enters only in the evolution of a trajectory from an initial point in phase space $\qc(t|\qc_0)$.
The statistics of the initial field distribution are encoded entirely in $\mathcal{P}(\qc_0,t_0)$.
{\bf Brief description of the relationship between the one-particle PDF and the initial field statistics.  Give general description and also special case of a Gaussian Random Field.}
We are thus free to describe the dynamics either in terms of individual lattice trajectories (the Langrangian approach, or Langevin approach for stochastic dynamics),
or else in terms of the processed probability distributions (the Eulerian approach, or Fokker-Planck approach for stochastic dynamics).

From~\eqref{eqn:probability-transport}, we immediately have
\begin{equation}
  \mathcal{P}(\qc_f,t) = \int_{\qc(t|\qc_0)=\qc_f}d\qc_0 \left|\frac{\partial\qc(t|\qc_0)}{\partial\qc_0} \right|^{-1} \mathcal{P}(\qc_0,t_0)
\end{equation}
In general, $\qc_0 = \qc^{-1}(t|\qc_f)$ may have multiple solutions as a function of $\qc_0$, leading to a sum over multiple ``streams'' in the above integral (or an integral over a continuous parameter in the case that the solutions are not discrete.
{\bf Show some special cases}

Usually we are interested in the projection into some lower dimensional space $\vec{\mathcal{O}}(\qc)$ rather than the full phase-space dynamics of the system.
{\bf or else the distribution in nonCanonical coordinates, or even in some higher dimensional space ...}
The corresponding probability density in the reduced space is
\begin{equation}
  \mathcal{P}_{\vec{\mathcal{O}}}(\vec{\mathcal{O}}_f,t) = \int d\qc \delta(\vec{\mathcal{O}}_f-\vec{\mathcal{O}}(\qc))) \mathcal{P}(\qc,t)
\end{equation}
{\bf Go through the details of this}

If we consider flows that preserve phase-space volume (such as those in a canonical Hamiltonian system), then the Jacobian determinant is constant in time.
However, if we consider projections down to lower dimensional spaces, choices of non-canonical variables, or ``cold'' initial conditions restricted to some submanifold of phase space,
then the mapping from initial positions to final phase space locations along the trajectories may develop singularities (known as caustics {\bf catastrophes?}) where the Jacobian determinant is $0$.  This results in a singularity in the final probability distribution function in the projected subspace, which can provide the dominant contribution to expectation values.
Physically, a caustic occurs from a focussing of trajectories in the reduced phase space.  In the original field theory, this corresponds to a collection of neighbouring lattice sites all undergoing nearly identical evolutions, and a corresponding spatially coherent feature forming.

For highly mixed trajectories, we expect initial bundles that avoid caustics to average out and produce the mean expansion history.  However, for bundles centered on special initial values that result in a caustic in $\ln a$, many of the individual trajectories will add coherently into the smoothed expansion history resulting in a spike in $\zeta$. {\bf Figure out where this idea belongs.}

\subsection{General Description of Trajectory Dynamics, Singular Embeddings of Lagrangian Submanifolds, etc.}

\subsection{Geodesic Deviation Equation and Caustic Formation}
To identify the formation of caustics, it is convenient to consider the time-evolution of the deviation between two infinitesimally separated trajectories
\begin{equation}
  \delta\qc(t|\qc_0) \equiv \qc(t|\qc_0+\delta\qc_0) - \qc(t|\qc_0)
\end{equation}
which can be expanded for infinitesimal initial perturbations $\delta\qc_0$ as
\begin{equation}
  \delta\qc(t|\qc_0) = \frac{\partial \qc}{\partial \qc_0}\bigg|_{\qc(t|\qc_0)}\cdot\delta\qc_0 \equiv {\bf D}\cdot\delta\qc_0
\end{equation}
or in component notation
\begin{equation}
  \delta q_i(t|\qc_0) = \frac{\partial q_i}{\partial q_{0,j}}\bigg|_{\qc(t|\qc_0)}\delta q_{0,j} \equiv D_{ij}\delta q_{0,j}
\end{equation}
{\bf Ok, if we have a zero direction, presumably this means we should actually go to next order in the perturbations.  Similar to checking for marginal operators in renormalisation}
For an autonomous system, the equations of motion for the phase space vector are given by
\begin{equation}
  \dot{\qc} = {\bf F}(\qc) \, .
\end{equation}
{\bf We could remove zeta as a dependent variable and make it non-autonomous?  Then we have the view of carrying zeta around with us, kind of like an entropy associated with the trajectory.}
In the limit of infinitesimal initial perturbations we have
\begin{equation}
  \delta\dot{\qc} = \dot{\qc}(t|\qc_0+\delta\qc_0) - \dot{\qc}(t|\qc_0) = {\bf F}(\qc(t|\qc_0+\delta\qc_0)) - {\bf F}(\qc(t|\qc_0)) = \frac{\partial {\bf F}}{\partial \qc}\cdot\delta\qc_0
\end{equation}

From this, we obtain an equation of motion for the deviation tensor
\begin{equation}
  \dot{D} = \frac{\partial{\bf F}}{\partial\qc}\cdot {\bf d} \equiv {\bf T}\cdot {\bf D}
\end{equation}
or in a coordinate basis defined by a choice of canonical coordinates
\begin{equation}
  \dot{D}_{ij} = \frac{\partial F_i}{\partial q_l}D_{lj}\equiv T_{il}D_{lj}
\end{equation}

\subsection{Specialisation to Hamiltonian Systems in Canonical Coordinates}
For a Hamiltonian system expressed in terms of a canonical set of phase space coordinates, we can divide our $2N$ phase space coordinates into $N$ generalised position and $N$ generalised momentum coordinates $\qc = \left({\bf r},{\bf \pi}\right)$ with corresponding equations of motion
\begin{equation}
  \frac{d {\bf r}}{dt} = \frac{\partial\mathcal{H}}{\partial {\bf \pi}} \qquad \frac{d{\bf \pi}}{dt} = -\frac{\partial \mathcal{H}}{\partial {\bf r}}
\end{equation}
so we see the tidal tensor is closely related to the Hessian of the Hamiltonian function in canonical coordinate space
\begin{equation}
  T_{ij} = J_{ik} \frac{\partial \mathcal{H}}{\partial q_k\partial q_j} \qquad J = \left[\begin{array}{cc} 0 & \mathbb{I} \\ -\mathbb{I} & 0 \end{array}\right]
\end{equation}
Using the identity $\ln\mathrm{det}D = \mathrm{Tr}\ln D$, is is straightforward to see that the determinant of $D$ is time-independent provided $D$ is invertible initially.
This just corresponds to the preservation of phase-space volume by the Hamiltonian flow.

\subsection{Projection onto Reduced Initial Condition Phase Space}
\begin{itemize}
\item Include some narrow Gaussian width in various IC directions
\item Introduce bred vectors, etc. as the important thing to pick out relevant directions
\end{itemize}

\subsection{Additional Things to Mention}
\begin{itemize}
\item Bred vectors
\item Equivalent of Zel'dovich approximation?
\item zero-mode instability is important for this picture
\item Analytic calculation of caustics in phase space?
\end{itemize}

\subsection{Brief reminder of stuff: To be removed}
Several regimes to consider
\begin{itemize}
\item Classical motion with stochastic noise term.  Occurs while inflating and subhorizon modes cross the horizon.
\item This stage generates the isocurvature mode that allows for the chaotic motion to be realised
\item Classical decoupled trajectories with no noise term.  After inflation ends, but before onset of strong inhomogeneities from preheating.
\item Post-inflation can extend the decoupled trajectories approximation into the subhorizon regime, in the limit that the zero mode is unstable (and continuity for k small, so those modes are also unstable)
\item Fully mode-mode coupled inhomogeneous evolution.

\end{itemize}


\section{Ballistic Dynamics During Preheating}

Things to understand
\begin{itemize}
\item What determines the width of the packet that enters into the arm?  This will somehow determine the strength and width of the spike.
\item The width of the packet in field space that goes in the arm can be related to the spatial size.  i.e. fraction of Hubble volume going into the arm.  This presumably tells us something about how the height of the spike scales and their width.
\end{itemize}

\subsection{Density Perturbations From Caustic Formation}
Stages of post-inflation dynamics.  These may or may not occur in a given model.
\begin{itemize}
\item Initially have Floquet theory.  At each passage through (minimum?) the length of the string increases, while maintaining roughly constant density.
\item During bouncing off hyperbolic walls, the field trajectories can split.  This leads to peak in the density of the string as a function of ``invariant'' distance.
\item Eventually get folding of trajectories.
\item It's important to have an instability in the zero mode, or there will be no splitting of trajectories.
\end{itemize}

The post-inflationary dynamics passes through a series of distinct stages.
Our focus here is on a specific model, but the general features we describe can be ubiquitous for a variety of potential minima.
The key feature for our analysis is the exponential divergence of neighbouring trajectories during the oscillations of the field around the potential minimum.
This requires an instability of the zero-mode of the 
In the initial stages of post-inflationary evolution, the field undergoes damped oscillations around the minimum of the potential.

For our specific model, the field initially oscillates primarily in the $\phi$ direction, and the zero-mode of the $\chi$ field (equivalently the angular direction for the small $\chi$ limit) experiences a parametric instability described by Floquet theory.
For certain choices of $g^2/\lambda$, the zero mode of $\chi$ is unstable, leading to a rotation of the line of oscillation for the individual trajectories at the bottom of the well.
The instability also causes the individual trajectories to diverge from each other, resulting in an elongation of the phase string.
{\bf This elongation is more or less uniform, and no sharp features are imprinted onto the density during this phase}.

The Floquet instability continues until the field trajectories have rotated enough to probe the hyperbolic parts of the potential isocurvature lines at the bottom of the potential.
Once this occurs, the linearised Floquet analysis breaks down and we enter the chaotic billiard dynamics regime.
The reflection of the individual lattice sites from the potential barriers acts as a softened version of the reflection of a billiard ball from the rails of a billiard table.
It is well known that billiard motion on a variety of non-rectangular tables (and in particular tables with hyperbolic wall segments) results in focussing and splitting of trajectories along the table.
In phase space, this corresponds to the folding and wrapping of the string, with the caustics resulting from the projection back to $\phi,\chi$ space corresponding to {\bf write this more eloquently}.

From our dynamical simulations, we can directly observe the build-up of these foldings and the resulting trajectory focussing and caustic formation.
This process is illustrated in~\figref{fig:}.

\begin{itemize}
\item Bounces off of walls now begin to split the trajectories at specific points, rather than uniformly along the entire string
\item Repeated bounces lead to folding of the string due to these splittings (ie. since it's inhomogeneous along the strong)
\item The folds project into caustics in the $\ln a$ plane
\item For $\bar{\chi}$ located near such a caustic value for $\chi_0$, there is a singular contribution to $\zeta$ when averaging over a horizon volume.  This occurs because many of the trajectories will have the same expansion history near the caustic (there is a bunching up in the potential).
\item For $\bar{\chi}$ away from such caustics, phase mixing in the bottom of the well smooths out the overall expansion history that that of a fluid with $w=1/3$ for the potential with dimensionless couplings that we use here.
\end{itemize}
{\bf Floquet phase of zero mode relies on the conformal rescaling.  However, the bifurcation dynamics does not.}

\subsection{Extension to Other Models}
The inflationary dynamics generating the density perturbations in the CMB can be decoupled from the potential shape during the post-inflation oscillations.
This will be generic if inflation occurs on a random potential such as in the landscape paradigm.
In particular, the inflaton may traverse a region with flat directions during inflation which are sufficient to generate superhorizon isocurvature modes which will act as a modulator for the preheating dynamics.

{\bf Need generation of the isocurvature modes, which act as modulators post inflation.}
{\bf Question: If I generate isocurvature modes then snap down to a trough, do the superhorizon fluctuations remain there?}

For example, if we restrict to two-field models we may consider the following parameterisation of the minimum
\begin{equation}
  V(\rho,\theta) = \sum_{m} R_m(\rho)\cos(m\theta + \alpha_m)
\end{equation}
with
\begin{equation}
  \phi_1 = \rho\cos\theta \qquad \phi_2 = \rho\sin\theta
\end{equation}
As an example, such potentials arise naturally if we consider Taylor expanding around the minimum
\begin{equation}
  V(\phi_{min}+\delta\phi) = V(\phi_{min})
\end{equation}
with each additional derivative term in $V$ contributing an additional Chebyshev mode to our expansion.
However, the higher terms in the Taylor expansion also include contributions to the lower Chebyshev modes.
Due to the nice numerical properties of choosing non-monomial potentials as a basis, it is instead nicer to work with a Chebyshev expansion.

\subsection{Include Critical Lines Flowing Into the Minimum}

\begin{figure}
  \caption{Various phase space projections of the dynamics.}
\end{figure}

\begin{figure}
  \caption{Show ``length'' of the string in phase space (under some definition) as a function of time.  This will play nicely with the density.  The question is what to use for density measurements.  Perhaps it's useful to show the length projected into a number of different planes?}
\end{figure}

\begin{figure}
  \caption{Show splitting of trajectories according to which arm they enter.}
\end{figure}

\begin{figure}
  \caption{Poincare slices}
\end{figure}

\begin{figure}
  \caption{Nice picture of caustic formation.}
\end{figure}

\begin{figure}
  \caption{Show trajectory of bifurcation points (ie caustics on initial string) in the potential}
\end{figure}

\begin{figure}
  \caption{Show the effective potential.}
\end{figure}

\section{Density Perturbations from Features in the Inflationary Potential}
We have presented a generic mechanism for generating adiabatic density perturbations from the nonlinear dynamics of oscillating scalar fields after inflation.
The ballistic approach used here relies on the presence of instabilities in the zero-mode dynamics of the field,
a mechanism to generate initial field fluctuations that are able to act as a modulator for the posti inflation dynamics,
and {\bf way to convert chaotic behaviour and caustics into actual differences in expansion history}.
However, similar considerations naturally apply to inflationary dynamics themselves.
In this case, the well-known picture of stochastic inflation {\bf yadda yadda, blah blah}.
{\bf Expand on how to apply this viewpoint to stochastic inflation as well.}

\section{CMB Maps from Generalised Local NonGaussianity}
\begin{itemize}
\item Description of the form of nonGaussianity produced by these models.
\item Description of how to do marginalisation from coarse-grained distribution (ie. smoothed over a Hubble at the end of inflation) to pixels in CMB of LSS
\item T and E maps from generalised local nonGaussianity
\end{itemize}

For an observer,  the primordial comoving curvature fluctuations  $\zeta$ can be expanded with spherical harmonics
\begin{equation}
  \zeta(r, \mathbf{n}) = \sum_{lm} Y_{lm}(\mathbf{n})\zeta_{lm}(r),
\end{equation}
where $r$ is the distance from the observer and $\mathbf{n} = (\theta, \phi)$ the line of sight direction. For linear projection, the projected temperature harmonic coefficient $a_{lm}^T$ is a superposition of $\zeta_{lm}(r)$ with appropriate weights $F_l^T(r)$. In this subsection we derive $F_l^T(r)$ and describe the numeric techniques to compute it.

 For a comoving wave number $\mathbf{k}$, the Fourier mode $\zeta_{\mathbf{k}}$ can be written as
\begin{equation}
  \zeta_{\mathbf{k}} = 4\pi \int_0^\infty r^2dr\sum_{lm}\zeta_{lm}(r) (-i)^l j_l(kr) Y_{lm}(\hat{\mathbf{k}})\, , \label{eq:zetakint}
\end{equation}
where we have used the identity
\begin{equation}
  e^{i\mathbf{k}\cdot\mathbf{n}r} = 4\pi \sum_{lm}i^l j_l(kr) Y^*_{lm}(\hat{\mathbf{k}}) Y_{lm}(\mathbf{n}) \, , \label{eq:eikx_expand}
\end{equation}
and orthogonality of $Y_{lm}$ functions.


The temperature field can be integrated along the line of sight
\begin{equation}
T({\mathbf{n}}) = \int S_T(\mathbf{k}, r') e^{i \mathbf{k}\cdot\mathbf{n} r'}\zeta_{\mathbf{k}}\frac{d^3k}{(2\pi)^3}, \label{eq:Tint1}
\end{equation}
where $S(\mathbf{k}, r')$ is the source function that can be computed using a Boltzmann code.

Plugging \eqref{eq:zetakint} and \eqref{eq:eikx_expand} into \eqref{eq:Tint1}, we obtain the desired expression
\begin{equation}
a_{lm}^T = \int_0^\infty r^2dr \zeta_{lm}(r) F^T_l(r), \label{eq:Tint2}
\end{equation}
where
\begin{equation}
F^T_l(r) = \frac{2}{\pi} \int k^2dk j_l(kr) \int dr' S_T(k, r') j_l(kr') \, . \label{eq:FTint}
\end{equation}

Although it seems to be profitable to convert the 3D integral \eqref{eq:Tint2} to 2D by first integrating over $r'$ in \eqref{eq:FTint}, practically it is not the case. The method we adopt here is to integrate along $k$ direction first. Because the source $S(k, r')$ is relatively smooth, it acts as a low-pass filter on the product of bessel functions $j_l(kr)j_l(kr')$. The numeric accuracy can be greately enhanced by doing a WKB expansion of $j_l(kr)j_l(kr')$ and keeping only the low frequency part of it. The high frequency part of $j_l(kr)j_l(kr')$ adds noises that are amplified by $k^2$ factor at large $k$. These noises, however, are fast-oscillating cosine waves and have negligible contribution to the total integral. The robustness of this method can be tested in two ways. First, we can vary the frequency cutoff for the low-pass filtering and observe the stability of the result. Secondly, we can compare the realized CMB power spectrum $C_l^{TT}$ with its theoretical expectation. 

The $\chi$ to $\zeta$ mapping is defined for a given scale $L$. The $L^3$-volume averaged $\bar{\zeta}|_{L^3}$ is a function of the averaged $\bar{\chi}|_{L^3}$.
\begin{equation}
\bar{\zeta}\vert_{L^3}  = f\left(\bar{\chi}\vert_{L^3}\right)\, .
\end{equation}

The random Gaussian $\chi$ field can be realized in discrtized shells with known covariance matrix:
\begin{eqnarray}
  \langle \chi^*_{lm}(r) \chi_{l'm'}(r')\rangle &=& \frac{\pi^2}{2^{2-n_s}}  \delta_{ll'}\delta_{mm'} \mathcal{P}_S\left(\frac{2}{r+r'}\right) \left(\frac{2\max(r,r')}{r+r'}\right)^{1-n_s} \left(\frac{\min(r, r')}{\max(r,r')}\right)^l\\
 &&  \times \frac{\Gamma\left(l+\frac{n_s-1}{2}\right)}{\Gamma\left(l+\frac{3}{2}\right)\Gamma\left(4-n_s\right)} {\,_2F_1\,}\left( \frac{n_s}{2}-1,\, l+\frac{n_s-1}{2},\, l+ \frac{3}{2},\, \left(\frac{\min(r, r')}{\max(r,r')}\right)^2\right)     \nonumber \, ,
\end{eqnarray}
where $\mathcal{P}_S \propto k^{n_s-1}$ is the power spectrum of $\chi$.  The function $\,_2F_1$ is the ordinary hypergeometric function and $\Gamma$ the Gamma function.

At high resolution ($\ell \gtrsim 1000$) CMB lensing starts to kick in and linear transfer is not sufficiently accurate. We restrict the discussion in this paper to angular scales $\ell \lesssim 1000$ and leave the sophistication beyond linear transfer to our future work.




\section{Entropy Generation and the Curvature Perturbation}
The evolution of the large-scale comoving curvature perturbation can be related to the production of entropy in the coarse-grained system.
From the second law of thermodynamics, we have
\begin{equation}
  d\rho + (\rho+P)d\ln V = \frac{TdS}{V}
\end{equation}
and along individual ballistic trajectories, comoving conservation of stress-energy enforces
\begin{equation}
  \dot{\rho} + 3H(\rho+P) = 0
\end{equation}
which demonstrates that the entropy is conserved along individual ballistic trajectories.
However, we are interesting in the comoving curvature perturbation averaged over spatial regions of order the local Hubble.
This coarse-graining requires us to average over many individual trajectories in order to obtain the relevant large-scale perturbation.
Let's denote quantities averaged over a spatial region $\Omega$ by
\begin{equation}
  \mathcal{O}_\Omega = V^{-1}\int_\Omega d^3x a^3(x)\mathcal{O}(x) \qquad V \equiv \int_\Omega d^3x a^3(x)
\end{equation}
where $x$ are comoving coordinates which we can think of as labelling the geodesic motion of each individual ballistic trajectory.

{\bf What happens if we allow for anisotropy (while enforcing homogeneity?)}
\begin{equation}
  \frac{dS}{N} = \beta3\frac{\rho + P}{\bar{n}} d\zeta
\end{equation}

\subsection{Some Interesting Musings}
First consider a large volume of space, denoted by $\Omega_{UL}$.
Define a probability distribution $Q_{UL}(x) = a^3(x) / V_{UL}$ where $V_{UL} = \int_{\Omega_{UL}}d^3x a^3(x)$ is the volume of some superhorizon region encompassing at least our current Hubble volume.
This choice corresponds to weighting individual lattice sites by their relative increase in volume from some initial surface where they all have uniform size.
Now consider a partition of $\Omega_{UL}$ which we denote $\Omega = \{\Omega_i\}$ with $\cup_i \Omega_i = \Omega$ and $\Omega_i \cap \Omega_j = \delta_{ij}\Omega_i$.
This should be viewed as a crude approximation to performing a spatial averaging by convolution with a window function (which we will return to shortly).
Define a coarse-grained probability distribution associated with the partition $\Omega$ by $Q_\Omega(x) = V_i / V_{UL}$ where $i$ is chosen by the condition $x \in \Omega_i$.
The Kullback-Leibler divergence of $Q_\Omega$ from $Q_{UL}$ is
\begin{equation}
  D(Q_{UL} \parallel Q_\Omega) = \int_{\Omega_{UL}} d^3x Q_{UL}\ln\left(\frac{Q_{UL}}{Q_\Omega}\right) = \sum_i \left(\frac{V_i}{V_{UL}}\right)\left[\frac{3}{V_i}\int_{\Omega_i}d^3xa^3\ln a -\ln V_i\right]
\end{equation}

We can also define normalised probabilities on each subset in the partition $Q_i = a^3 / V_i$.
\begin{equation}
  \frac{1}{V_i}\int_{\Omega_i} d^3x a^3\ln a = \frac{1}{3} \int d^3x Q_i\ln Q_i + \frac{1}{3}\ln V_i
\end{equation}
where we've defined
\begin{equation}
  V_i = \int_{\Omega_i} d^3x a^3 \, .
\end{equation}
Therefore, the perturbation to $\zeta$ relative to what it would have been for a completely homogeneous evolution (with each trajectory experiencing the same expansion history) is encoded by the Shannon entropy associated with weighting individual trajectories by their overall volume expansion.

However, we may also wish to compare $\ln a$ in $\Omega$ to the average expansion in the ultra-large volume instead.
\begin{equation}
  \delta\zeta \equiv \langle \ln a \rangle_\Omega - \langle \ln a \rangle_{V_{UL}}
\end{equation}

\end{document}
