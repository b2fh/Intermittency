\documentclass{revtex4}

\usepackage{amsmath,amssymb}
\usepackage{graphicx}

\begin{document}

\section{Outline}
\begin{itemize}
\item Introduction - Inflation is awesome.  Standard separate universe picture during inflation.  We extend this into the post-inflation regime.  Phenomena resulting from averaged billiard trajectories.  Natural mechanism to produce spatially intermittent signatures (from a subset of the initial packet of trajectories crossing a feature in the potential, from caustic formation, ect)
\item There's a nice story 
\end{itemize}

\section{Introduction}
BLAH BLAH BLAH BLAH BLAH BLAH BLAH BLAH

\section{Multiscale Hierarchy and Positions as Billiards}
Outline the basic scales in the problem and indicate that we have a multiple hierarchies to consider.
Also explain how we will basically view each spatial position as some individual ballistic trajectory.
This generalises the philosophy behind the separate universe approximation.
When going from the smaller length scales back up to larger length scales in the hierarchy, convolutions must be done.
This allows for entropy generation via coarse-graining.
The complicated motion of the trajectories then allows for the coarse-graining to in some sense operate very differently in different spatial locations.
This leads to $\zeta$ perturbations and a relationship to entropy production.

\begin{figure}
  \includegraphics[width=0.9\linewidth]{{{multiscale_isocon}}}
  \caption{Multiscale structure of the isocurvature mode that acts as a modulator for the preheating dynamics.}
\end{figure}


{\bf Write this section}

\section{Entropy Generation and the Curvature Perturbation}
The evolution of the large-scale comoving curvature perturbation can be related to the production of entropy in the coarse-grained system.
From the second law of thermodynamics, we have
\begin{equation}
  d\rho + (\rho+P)d\ln V = \frac{TdS}{V}
\end{equation}
and along individual ballistic trajectories, comoving conservation of stress-energy enforces
\begin{equation}
  \dot{\rho} + 3H(\rho+P) = 0
\end{equation}
which demonstrates that the entropy is conserved along individual ballistic trajectories.
However, we are interesting in the comoving curvature perturbation averaged over spatial regions of order the local Hubble.
This coarse-graining requires us to average over many individual trajectories in order to obtain the relevant large-scale perturbation.
Let's denote quantities averaged over a spatial region $\Omega$ by
\begin{equation}
  \mathcal{O}_\Omega = V^{-1}\int_\Omega d^3x a^3(x)\mathcal{O}(x) \qquad V \equiv \int_\Omega d^3x a^3(x)
\end{equation}
where $x$ are comoving coordinates which we can think of as labelling the geodesic motion of each individual ballistic trajectory.

{\bf What happens if we allow for anisotropy (while enforcing homogeneity?)}
\begin{equation}
  \frac{dS}{N} = \beta3\frac{\rho + P}{\bar{n}} d\zeta
\end{equation}

\subsection{Some Interesting Musings}
First consider a large volume of space, denoted by $\Omega_{UL}$.
Define a probability distribution $Q_{UL}(x) = a^3(x) / V_{UL}$ where $V_{UL} = \int_{\Omega_{UL}}d^3x a^3(x)$ is the volume of some superhorizon region encompassing at least our current Hubble volume.
This choice corresponds to weighting individual lattice sites by their relative increase in volume from some initial surface where they all have uniform size.
Now consider a partition of $\Omega_{UL}$ which we denote $\Omega = \{\Omega_i\}$ with $\cup_i \Omega_i = \Omega$ and $\Omega_i \cap \Omega_j = \delta_{ij}\Omega_i$.
This should be viewed as a crude approximation to performing a spatial averaging by convolution with a window function (which we will return to shortly).
Define a coarse-grained probability distribution associated with the partition $\Omega$ by $Q_\Omega(x) = V_i / V_{UL}$ where $i$ is chosen by the condition $x \in \Omega_i$.
The Kullback-Leibler divergence of $Q_\Omega$ from $Q_{UL}$ is
\begin{equation}
  D(Q_{UL} \parallel Q_\Omega) = \int_{\Omega_{UL}} d^3x Q_{UL}\ln\left(\frac{Q_{UL}}{Q_\Omega}\right) = \sum_i \left(\frac{V_i}{V_{UL}}\right)\left[\frac{3}{V_i}\int_{\Omega_i}d^3xa^3\ln a -\ln V_i\right]
\end{equation}

We can also define normalised probabilities on each subset in the partition $Q_i = a^3 / V_i$.
\begin{equation}
  \frac{1}{V_i}\int_{\Omega_i} d^3x a^3\ln a = \frac{1}{3} \int d^3x Q_i\ln Q_i + \frac{1}{3}\ln V_i
\end{equation}
where we've defined
\begin{equation}
  V_i = \int_{\Omega_i} d^3x a^3 \, .
\end{equation}
Therefore, the perturbation to $\zeta$ relative to what it would have been for a completely homogeneous evolution (with each trajectory experiencing the same expansion history) is encoded by the Shannon entropy associated with weighting individual trajectories by their overall volume expansion.

However, we may also wish to compare $\ln a$ in $\Omega$ to the average expansion in the ultra-large volume instead.
\begin{equation}
  \delta\zeta \equiv \langle \ln a \rangle_\Omega - \langle \ln a \rangle_{V_{UL}}
\end{equation}

\section{Ballistic Trajectories and the Spatially Decoupled Limit}
Several regimes to consider
\begin{itemize}
\item Classical motion with stochastic noise term.  Occurs while inflating and subhorizon modes cross the horizon.
\item This stage generates the isocurvature mode that allows for the chaotic motion to be realised
\item Classical decoupled trajectories with no noise term.  After inflation ends, but before onset of strong inhomogeneities from preheating.
\item Post-inflation can extend the decoupled trajectories approximation into the subhorizon regime, in the limit that the zero mode is unstable (and continuity for k small, so those modes are also unstable)
\item Fully mode-mode coupled inhomogeneous evolution.
\end{itemize}

\section{Creation of Spatially Intermittent Features from Ballistic Trajectory Dynamics} 
Things to understand
\begin{itemize}
\item What determines the width of the packet that enters into the arm?  This will somehow determine the strength and width of the spike.
\item The width of the packet in field space that goes in the arm can be related to the spatial size.  i.e. fraction of Hubble volume going into the arm.  This presumably tells us something about how the height of the spike scales and their width.
\end{itemize}

\subsection{Density Perturbations From Caustic Formation}
Stages of post-inflation dynamics.  These may or may not occur in a given model.
\begin{itemize}
\item Initially have Floquet theory.  At each passage through (minimum?) the length of the string increases, while maintaining roughly constant density.
\item During bouncing off hyperbolic walls, the field trajectories can split.  This leads to peak in the density of the string as a function of ``invariant'' distance.
\item Eventually get folding of trajectories.
\item It's important to have an instability in the zero mode, or there will be no splitting of trajectories.
\end{itemize}

The post-inflationary dynamics passes through a series of distinct stages.
Our focus here is on a specific model, but the general features we describe can be ubiquitous for a variety of potential minima.
The key feature for our analysis is the exponential divergence of neighbouring trajectories during the oscillations of the field around the potential minimum.
This requires an instability of the zero-mode of the 
In the initial stages of post-inflationary evolution, the field undergoes damped oscillations around the minimum of the potential.

For our specific model, the field initially oscillates primarily in the $\phi$ direction, and the zero-mode of the $\chi$ field (equivalently the angular direction for the small $\chi$ limit) experiences a parametric instability described by Floquet theory.
For certain choices of $g^2/\lambda$, the zero mode of $\chi$ is unstable, leading to a rotation of the line of oscillation for the individual trajectories at the bottom of the well.
The instability also causes the individual trajectories to diverge from each other, resulting in an elongation of the phase string.
{\bf This elongation is more or less uniform, and no sharp features are imprinted onto the density during this phase}.

The Floquet instability continues until the field trajectories have rotated enough to probe the hyperbolic parts of the potential isocurvature lines at the bottom of the potential.
Once this occurs, the linearised Floquet analysis breaks down and we enter the chaotic billiard dynamics regime.
The reflection of the individual lattice sites from the potential barriers acts as a softened version of the reflection of a billiard ball from the rails of a billiard table.
It is well known that billiard motion on a variety of non-rectangular tables (and in particular tables with hyperbolic wall segments) results in focussing and splitting of trajectories along the table.
In phase space, this corresponds to the folding and wrapping of the string, with the caustics resulting from the projection back to $\phi,\chi$ space corresponding to {\bf write this more eloquently}.

From our dynamical simulations, we can directly observe the build-up of these foldings and the resulting trajectory focussing and caustic formation.
This process is illustrated in~\figref{fig:}.

\begin{itemize}
\item Bounces off of walls now begin to split the trajectories at specific points, rather than uniformly along the entire string
\item Repeated bounces lead to folding of the string due to these splittings (ie. since it's inhomogeneous along the strong)
\item The folds project into caustics in the $\ln a$ plane
\item For $\bar{\chi}$ located near such a caustic value for $\chi_0$, there is a singular contribution to $\zeta$ when averaging over a horizon volume.  This occurs because many of the trajectories will have the same expansion history near the caustic (there is a bunching up in the potential).
\item For $\bar{\chi}$ away from such caustics, phase mixing in the bottom of the well smooths out the overall expansion history that that of a fluid with $w=1/3$ for the potential with dimensionless couplings that we use here.
\end{itemize}
{\bf Floquet phase of zero mode relies on the conformal rescaling.  However, the bifurcation dynamics does not.}


\subsection{Extension to Other Models}
The inflationary dynamics generating the density perturbations in the CMB can be decoupled from the potential shape during the post-inflation oscillations.
This will be generic if inflation occurs on a random potential such as in the landscape paradigm.
In particular, the inflaton may traverse a region with flat directions during inflation which are sufficient to generate superhorizon isocurvature modes which will act as a modulator for the preheating dynamics.

{\bf Need generation of the isocurvature modes, which act as modulators post inflation.}
{\bf Question: If I generate isocurvature modes then snap down to a trough, do the superhorizon fluctuations remain there?}

For example, if we restrict to two-field models we may consider the following parameterisation of the minimum
\begin{equation}
  V(\rho,\theta) = \sum_{m} R_m(\rho)\cos(m\theta + \alpha_m)
\end{equation}
with
\begin{equation}
  \phi_1 = \rho\cos\theta \qquad \phi_2 = \rho\sin\theta
\end{equation}
As an example, such potentials arise naturally if we consider Taylor expanding around the minimum
\begin{equation}
  V(\phi_{min}+\delta\phi) = V(\phi_{min})
\end{equation}
with each additional derivative term in $V$ contributing an additional Chebyshev mode to our expansion.
However, the higher terms in the Taylor expansion also include contributions to the lower Chebyshev modes.
Due to the nice numerical properties of choosing non-monomial potentials as a basis, it is instead nicer to work with a Chebyshev expansion.

\subsection{Include Critical Lines Flowing Into the Minimum}


\begin{figure}
  \caption{Various phase space projections of the dynamics.}
\end{figure}

\begin{figure}
  \caption{Show ``length'' of the string in phase space (under some definition) as a function of time.  This will play nicely with the density.  The question is what to use for density measurements.  Perhaps it's useful to show the length projected into a number of different planes?}
\end{figure}

\begin{figure}
  \caption{Show splitting of trajectories according to which arm they enter.}
\end{figure}

\begin{figure}
  \caption{Poincare slices}
\end{figure}

\begin{figure}
  \caption{Nice picture of caustic formation.}
\end{figure}

\begin{figure}
  \caption{Show trajectory of bifurcation points (ie caustics on initial string) in the potential}
\end{figure}

\begin{figure}
  \caption{Show the effective potential.}
\end{figure}

\section{Density Perturbations from Features in the Inflationary Potential}
We have presented a generic mechanism for generating adiabatic density perturbations from the nonlinear dynamics of oscillating scalar fields after inflation.
The ballistic approach used here relies on the presence of instabilities in the zero-mode dynamics of the field,
a mechanism to generate initial field fluctuations that are able to act as a modulator for the posti inflation dynamics,
and {\bf way to convert chaotic behaviour and caustics into actual differences in expansion history}.
However, similar considerations naturally apply to inflationary dynamics themselves.
In this case, the well-known picture of stochastic inflation {\bf yadda yadda, blah blah}.
{\bf Expand on how to apply this viewpoint to stochastic inflation as well.}

\end{document}
