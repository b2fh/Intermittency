\section{CMB Maps from Generalised Local NonGaussianity}
\begin{itemize}
\item Description of the form of nonGaussianity produced by these models.
\item Description of how to do marginalisation from coarse-grained distribution (ie. smoothed over a Hubble at the end of inflation) to pixels in CMB of LSS
\item T and E maps from generalised local nonGaussianity
\end{itemize}

For an observer,  the primordial comoving curvature fluctuations  $\zeta$ can be expanded with spherical harmonics
\begin{equation}
  \zeta(r, \mathbf{n}) = \sum_{lm} Y_{lm}(\mathbf{n})\zeta_{lm}(r),
\end{equation}
where $r$ is the distance from the observer and $\mathbf{n} = (\theta, \phi)$ the line of sight direction. For linear projection, the projected temperature harmonic coefficient $a_{lm}^T$ is a superposition of $\zeta_{lm}(r)$ with appropriate weights $F_l^T(r)$. In this subsection we derive $F_l^T(r)$ and describe the numeric techniques to compute it.

 For a comoving wave number $\mathbf{k}$, the Fourier mode $\zeta_{\mathbf{k}}$ can be written as
\begin{equation}
  \zeta_{\mathbf{k}} = 4\pi \int_0^\infty r^2dr\sum_{lm}\zeta_{lm}(r) (-i)^l j_l(kr) Y_{lm}(\hat{\mathbf{k}})\, , \label{eq:zetakint}
\end{equation}
where we have used the identity
\begin{equation}
  e^{i\mathbf{k}\cdot\mathbf{n}r} = 4\pi \sum_{lm}i^l j_l(kr) Y^*_{lm}(\hat{\mathbf{k}}) Y_{lm}(\mathbf{n}) \, , \label{eq:eikx_expand}
\end{equation}
and orthogonality of $Y_{lm}$ functions.


The temperature field can be integrated along the line of sight
\begin{equation}
T({\mathbf{n}}) = \int S_T(\mathbf{k}, \chi) e^{i \mathbf{k}\cdot\mathbf{n} \chi}\zeta_{\mathbf{k}}\frac{d^3k}{(2\pi)^3}, \label{eq:Tint1}
\end{equation}
where $S(\mathbf{k}, \chi)$ is the source function that can be computed using a Boltzmann code.

Plugging \eqref{eq:zetakint} and \eqref{eq:eikx_expand} into \eqref{eq:Tint1}, we obtain the desired expression
\begin{equation}
a_{lm}^T = \int_0^\infty r^2dr \zeta_{lm}(r) F^T_l(r), \label{eq:Tint2}
\end{equation}
where
\begin{equation}
F^T_l(r) = \frac{2}{\pi} \int k^2dk j_l(kr) \int d\chi S_T(k, \chi) j_l(k\chi) \, . \label{eq:FTint}
\end{equation}

Although it seems to be profitable to convert the 3D integral \eqref{eq:Tint2} to 2D by first integrating over $\chi$ in \eqref{eq:FTint}, practically it is not the case. Because the source $S(k, \chi)$ is relatively smooth, it acts as a low-pass filter on the product of bessel functions $j_l(kr)j_l(k\chi)$. The numeric accuracy can be greately enhanced by doing a WKB expansion of $j_l(kr)j_l(k\chi)$ and keeping only the low frequency part of it. The high frequency part of $j_l(kr)j_l(k\chi)$ adds noises that are amplified by $k^2$ factor at large $k$. These noises, however, are fast-oscillating cosine waves and have negligible contribution to the total integral. The robustness of this method can be tested in two ways. First, we can vary the frequency cutoff for the low-pass filtering and observe the stability of the result. Secondly, we can compare the realized CMB power spectrum $C_l^{TT}$ with the theoretical expectation. 

At higher $\ell \gtrsim 1000$ CMB lensing starts to kick in and linear transfer is not sufficiently accurate. We restrict the discussion in this paper to angular scales $\ell \lesssim 1000$ and leave the sophistication beyond linear transfer to our future work.


