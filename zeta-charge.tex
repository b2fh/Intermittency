\section{Curvature Perturbation, Entropy Generation, and }
The evolution of the large-scale comoving curvature perturbation can be related to the production of entropy in the coarse-grained system.
From the second law of thermodynamics, we have
\begin{equation}
  d\rho + (\rho+P)d\ln V = \frac{TdS}{V}
\end{equation}
and along individual ballistic trajectories, comoving conservation of stress-energy enforces
\begin{equation}\label{eqn:rho-cons}
  \dot{\rho} + 3H(\rho+P) = 0
\end{equation}
which demonstrates that the entropy is conserved along individual ballistic trajectories.
It is convenient to define
\begin{equation}
  d\zeta = \frac{d\ln\rho}{3(1+w)} + \frac{1}{3}d\ln V
\end{equation}
so that along individual paths we have
\begin{equation}
  \zeta(t) \equiv \int_{t_i}^t dt \frac{1}{3(1+w)}\frac{d\ln\rho}{dt} + \frac{d\ln a}{dt} \, .
\end{equation}
From~\eqref{eqn:rho-cons}, we see that $\zeta$ is conserved along ballistic trajectories as long as the the trajectories do not interact with each other and gradient terms can be neglected.
However, when gradients in the field become important, this trajectory wise conservation is broken.
In a simplistic case where we assume the metric continues to be FRW, but allow the fields $\phi_i$ to be inhomogeneous, we have
\begin{equation}
  \frac{1}{3(1+w)}\frac{d\ln\rho}{dt} + \frac{d\ln a}{dt} = \sum_i\frac{\nabla\cdot(\dot{\phi_i}\nabla\phi_i)}{3a^2(\rho+P)}
\end{equation}
for the case of a collection of scalar fields with canonical kinetic term.
Beyond the ballistic approximation, the individual trajectories will be connected to each other via spatial gradients (and non-locally through the Hubble constraint).

When considering the effects of divergences between small scale trajectories on the overall large-scale expansion, it is convenient to remove the effects of the current being exchanged between the trajectories by writing
\begin{equation}
  \frac{d\zeta}{dt} = \frac{1}{a^2}\nabla\cdot\left(\frac{\dot{\phi}_i\nabla\phi_i}{3(\rho+P)}\right) + \frac{1}{3a^2}\frac{\dot{\phi}\nabla\phi}{\rho+P}\cdot\nabla\ln(\rho+P) \, .
\end{equation}
We therefore define the $\zeta$ current
\begin{equation}
  {\bf J}_{\zeta} \equiv \frac{\dot{\phi}\nabla\phi}{3(\rho+P)}
\end{equation}
so that
\begin{equation}
  \frac{d\zeta}{dt} = \nabla\cdot{\bf J}_\zeta + {\bf J}_\zeta\cdot\nabla\ln(\rho+P) \, .
\end{equation}
If we consider $\zeta$ averaged on a constant $t$ slice, we then find
\begin{equation}
  \frac{d\langle\zeta\rangle_V}{dt} = \frac{1}{V}\int (1+\ln(\rho+P)){\bf J}_\zeta\cdot d{\bf \Sigma} - \frac{1}{V}\int d{\bf x}\ln(\rho+P)\nabla\cdot{\bf J}_\zeta \, .
\end{equation}
If we further define the $\zeta$ potential by
\begin{equation}
  \nabla^2\Psi_\zeta = \nabla\cdot{\bf J}_\zeta
\end{equation}
then we can alternatively write the volume term as
\begin{equation}
  \frac{1}{V}\int d{\bf x}\ln(\rho+P)\nabla\cdot{\bf J}_\zeta = \frac{1}{V}\int d{\bf x}\Psi_\zeta \nabla^2\ln(\rho+P) \, .
\end{equation}

{\bf Include more general case.  Allow for projections into fluid rest frame, etc., figure out which is the best frame to be using for this analysis}
