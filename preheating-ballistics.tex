\section{Curvature Perturbations from Ballistic Dynamics During Preheating}
Let's now consider how the trajectory dynamics and development of caustics manifest in preheating, and eventually lead to the production of (non-Gaussian) curvature perturbations.
We focus on a specific model in this section, although from the discussion it should be evident that the generic features leading to the production of curvature perturbations will appear in a wide variety of models.
Ultimately, we want to compute the curvature perturbation $\zeta = \ln a|_{\rho}(\chi_i)$ subject to the constraint of some initial mean value
\begin{equation}
  \chi_i^{(H_e)} \equiv \langle \chi \rangle_{H_e^{-1}}
\end{equation}
where the spatial average is taken over a local Hubble volume on the end of inflation hypersurface.

In section~\ref{sec:} we presented calculations of $\langle \zeta | \chi_i \rangle_{H_e}$ for a specific preheating model using full three-dimensional lattice simulations.
We also presented some preliminary evidence that the detailed structure of the resultant curvature perturbations could be understood in terms of smoothed motions of bundles of decoupled ballistic trajectories evolving in the potential.
We will now expand on this observation and demonstrate that the production of curvature spikes from the preheating dynamics can be understood in terms of the formation of caustics in the dynamics of trajectories.

In the ballistic approximation, an $n$-field scalar field model interacting with gravity lives in a $2n+2$-dimensional phase space along with a single constraint $3H^2 = \rho$.
To illustrate the basic phenomena, we take our single-particle Lagrangian to be (up to a total derivative)
\begin{equation}
  L = \frac{a^2}{2}\dot{\phi}^2 + \frac{a^2}{2}\dot{\chi}^2 - a^4\frac{\lambda}{4}\left(\phi^4 + \frac{g^2}{\lambda}\phi^2\chi^2\right) - 3M_P^2\dot{a}^2 + 3M_P^2\partial_\tau(a\dot{a})
\end{equation}
For simplicity, we set $M_P=1$ in what follows.

It is convenient to choose $\varphi_1 = a\phi$ and $\varphi_2 = a\chi$ as canonical field variables, with corresponding canonical momenta $\Pi_1 = a\dot{\phi}$ and $\Pi_2=a\dot{\chi}$ where a dot represents a derivative with respect to conformal time $\tau$.
In terms of these variables, the Hamiltonian is
\begin{equation}
  \mathcal{H} = \frac{\Pi_1^2}{2} + \frac{\Pi_2^2}{2} + \left(\varphi_1\Pi_1 + \varphi_2\Pi_2\right)\frac{\Pi_a}{a} + \frac{\lambda}{4}\left(\varphi_1^4 + \frac{2g^2}{\lambda}\varphi_1^2\vartheta_2^2 \right) - \frac{\Pi_a^2}{12 M_P^2}
\end{equation}
with the constraint
\begin{equation}
  \mathcal{H} = 0
\end{equation}
These yield the following equations of motion
\begin{align}\label{eqn:eom_scale}
  \frac{\partial\varphi_i}{\partial\tau} &= \Pi_i + \frac{\Pi_a}{a}\varphi_i \\
  \frac{\partial\Pi_i}{\partial\tau}     &= -\frac{\dot{a}}{a}\Pi_i - \frac{\partial \tilde{V}}{\partial\varphi} \\
  \frac{\partial a}{\partial\tau}        &= -\frac{\Pi_a}{6M_P^2} \\
  \frac{\partial\Pi_a}{\partial\tau}     &= \frac{\Pi_a}{a^2}\sum_i\varphi_i\Pi_i
\end{align}
where $\tilde{V}(\vec{\varphi}) = a^{-4}V(\vec{\phi})$.
Alternatively, in unscaled field variables $\phi,\chi$ we have
\begin{equation}
  \mathcal{H} = \sum_i\frac{\pi_i^2}{2a^2} + a^4V(\vec{\phi}) - \frac{\Pi_a^2}{12M_P^2}
\end{equation}
and equations
\begin{align}\label{eqn:eom_noscale}
  \frac{\partial\phi_i}{\partial\tau} &= \frac{\pi_i^2}{a^2} \\
  \frac{\partial\pi_i}{\partial_\tau} &= -a^4 \partial_iV(\phi) \\
  \frac{\partial a}{\partial\tau}     &= -\frac{\Pi_a}{6M_P^2} \\
  \frac{\partial\Pi_a}{\partial\tau}  &= \sum_i\frac{\pi_i^2}{a^3} - a^3V \, .
\end{align}

{\bf Give a better description of initial conditions}
To connect to our lattice simulations, we consider the evolution of a collection of decoupled ballistic trajectories.
These trajectories are centered on the mean values of the fundamental fields in the corresponding lattice simulations, and each one receives a weight corresponding to the probability that the particular value would be realised amongst the subhorizon fluctuations.
The mean value is drawn from the distribution of superhorizon values $\chi_i$, and each ballistic trajectory is then meant to simulate the evolution of the fields on subhorizon scales.
Therefore, we can completely specify the relevant initial data .
For Gaussian subhorizon fluctuations relevant to the case of cosmological preheating, the subhorizon joint PDF can be obtained by the covariance matrix.
To emulate the lattice simulations, we then compute equivalent lattice averages of functions of the basic field variables by evaluating the expectation value over the initial field bundle
\begin{equation}
  \langle \mathcal{O} \rangle(\tau) = \sum_{i=1}^{n_{\rm traj}} P_i\mathcal{O}(\qc(\tau|\qc_{0,i}))
\end{equation}
For example, the extract $\zeta$, we compute $\langle a \rangle$ and $\langle H \rangle$.
We then obtain $\zeta$ by moving to the desired $\langle H \rangle$ slice and extracting $\ln\langle a \rangle$.
In the classical limit, the model we consider is conformally invariant, so that $a^2H = \Pi_a$ is constant when averaged over an oscillation of the fields at the bottom of the potential.
Since we wish to extract the curvature perturbation on fixed $H$ slices, it is therefore convenient to instead consider the smoothed value of the time averaged $\Pi_a$ in order to construct $\zeta$.
This approach should be valid in cases where the most unstable field mode is initially the zero-mode.
Although this is not completely generic, we will argue below that this occurs for a wide range of shapes near the local minima of the scalar field potential.

The computational gain from this approach is tremendous.
Not only can we evolve each of our effective lattice sites independently, allowing for trivial parallelisation, we only need to consider each value of the initial conditions once and then just adjust the initial probability.
In comparison, in a single lattice simulation, we must evolve a sufficient number of points that the initial PDF for the subhorizon degrees of freedom is properly sampled.
Therefore, even without scanning over the mean values within the cube, we must already evolve many copies of trajectories with the same initial conditions (or more precisely with a much finer spacing in initial field values sampling the PDF).
When the ballistic approximation holds, this amounts to solving for a huge amount of redundant information.

We solve the equations of motion~\eqref{eqn:eom_noscale} using a tenth order accurate Gauss-Legendre integrator~\cite{Butcher,Braden}.
This integration scheme is symplectic, preserving quadratic invariants of the action as well as the canonical symplectic 2-form for the Hamiltonian system.
As will be seen, this dynamical system exhibits chaotic behaviour for a range of coupling parameter $2g^2/\lambda$.
Therefore, the preservation of the phase-space structure by our integration scheme is of prime importance when attempting to compute averaged quantities over a cloud of initial trajectories,
since the chaotic motion would quickly destroy preservation of probabilities for an inferior integration scheme.
Due to the extreme accuracy of our integrator, we are able to preserve the Hamiltonian constraint $\mathcal{H}=0$ to machine precision along each individual ballistic trajectory.
{\bf Figure?  Can explicitly see the bit-flipping and corresponding random walk over time if I plot this.  Probably a good idea to also halve the time-step and compare trajectories.}

%In this sense, it is more natural to consider the equations for the longitudinal and transverse parts of the field {\bf Work out this and do the calculation}.
%An approximate, but less technically involved approach, is to instead consider the radial and angular parts of the field defined through
%\begin{align}
%  \phi &= \sigma\cos\theta \\
%  \chi &= \sigma\sin\theta
%\end{align}

\begin{figure}
  \caption{Caustic plot in $\Pi_a = a^2H$.  On the left show the ballistic trajectory one (with appropriate smoothings), on the right show the full lattice simulation one.}
\end{figure}
{\bf Explain the structure in this caustic plot.  The periodicity is easy to understand.  Obtaining the basic structural block requires numerics.  We should also be able to get the scaling of the peaks widths, etc through some reasonably simple argument.  The existence of two distinct peaks occurs because there are 2 arms? (could easily check with another model with 3 arms say).  Substructure comes from usual period doubling and extreme coiling of the string as time evolves.}

We now present the evolution for a single period of the caustic structure in the initial values of $\ln\chi_i$.
For our example, this period is controlled by the Floquet exponent $\mu T_\phi$ for linear perturbations in $\chi$ around a trajectory with $\chi=0$.
As is well known, this problem can be reduced to the solution of the Lame equation~\cite{KofmanGreene} up to small corrections associated with the evolution of the scale factor $a$.
From the ballistic viewpoint, this stage is more naturally viewed as a rotation of trajectory's oscillation ``plane'' {\bf line?} in both the $(\phi,\chi)$ and $(\dot{\phi},\dot{\chi})$ planes.

\begin{figure}
  \caption{Length of the phase string as a function of time.  Here we have defined our metric on phase space as $ds^2 = d\varphi_1^2+d\varphi_2^2 + \Pi_1^2 + \Pi_2^2$.  In addition to the full length of the string, we also show }
\end{figure}

\begin{figure}
  \caption{Response of the phase space variables to small initial perturbations in the modulating field value $\delta\chi$.}
\end{figure}

\begin{figure}
  \includegraphics[width=0.49\linewidth]{{{phase-string-3d_t600}}}
  \includegraphics[width=0.49\linewidth]{{{phase-string-3d_t625}}}
  \caption{Three-Dimensional Projection of the initial phase string into the $(a\phi,a\chi,a\dot{\phi})$ cube.  The full phase string is shown in black.  The projections onto three remaining two-dimensional planes are shown in blue.  Also included for reference are isocontours on constant potential energy, energy in $\phi$ (with $\chi=0$) and ..., along with the isocontour corresponding to the initial value of the energy.  Since the momenta associated with the scale factor $a$ is conserved in time average, these initial isocontours can be viewed as roughly showing the projection of the constraint surface $\mathcal{H}=0$ into the appropriate planes.  From left to right we show the distribution at $\tau = $,$\tau = $, and $\tau = $.}
\end{figure}

\begin{figure}
  \caption{Entropy of projected PDF as a function of time?}
\end{figure}

\begin{figure}
  \caption{Distribution of $\ln a$ for a Gaussian bundle of trajectories centered around a spike (\emph{left}) and without a spike (\emph{right}).}
\end{figure}

{\bf To Do:  Fix the initial Hubble to be the same at the start of each inidividual trajectory?  Might remove one of the ugly splitting issues.}
