\section{Curvature Perturbations from Ballistic Dynamics During Preheating}
Let's now consider how the trajectory dynamics and development of caustics manifest in preheating, and eventually lead to the production of (non-Gaussian) curvature perturbations.
We focus on a specific model in this section, although from the discussion it should be evident that the generic features leading to the production of curvature perturbations will appear in a wide variety of models.
In order to compare with simulations, we want to compute the curvature perturbation $\zeta = \ln a|_{\rho}(\chi_i)$ (averaged over a volume of order the inverse Hubble volume at the end of inflation) subject to the constraint of some initial mean value
\begin{equation}
  \chi_i^{(H_e)} \equiv \langle \chi \rangle_{H_e^{-1}}
\end{equation}
where the spatial average is taken over a local Hubble volume on the end of inflation hypersurface.

In section~\ref{sec:} we presented calculations of $\langle \zeta | \chi_i \rangle_{H_e}$ for a specific preheating model using full three-dimensional lattice simulations.
We also presented some preliminary evidence that the detailed structure of the resultant curvature perturbations could be understood in terms of smoothed motions of bundles of decoupled ballistic trajectories evolving in the potential.
We will now expand on this observation and demonstrate that the production of curvature spikes from the preheating dynamics can be understood in terms of the formation of caustics in the dynamics of trajectories.

In the ballistic approximation, an $n$-field scalar field model interacting with gravity lives in a $2n+2$-dimensional phase space along with a single constraint $3H^2 = \rho$.
To illustrate the basic phenomena, we take our single-particle Lagrangian to be (up to a total derivative)
\begin{align}
  L &= \frac{a^2}{2}\dot{\phi}^2 + \frac{a^2}{2}\dot{\chi}^2 - a^4\frac{\lambda}{4}\left(\phi^4 + \frac{g^2}{\lambda}\phi^2\chi^2\right) - 3M_P^2\dot{a}^2 + 3M_P^2\partial_\tau(a\dot{a}) \\
    &= \frac{\dot{\varphi}_1^2}{2} + \frac{\dot{\varphi}_2^2}{2} +\frac{\dot{a}^2}{2a^2}\left(\varphi_1^2 + \varphi_2^2\right) - \left(\varphi_1\dot{\varphi}_1 + \varphi_2\dot{\varphi}_2\right)\frac{\dot{a}}{a} - \frac{\lambda}{4}\left(\varphi_1^4 + \frac{g^2}{\lambda}\varphi_1^2\varphi_2^2\right) - 3M_P^2\dot{a}^2 + 3M_P^2\partial_\tau(a\dot{a})
\end{align}
where we have defined $\varphi_i = a\phi_i$ and $M_P$ is the reduced Planck mass.

It is convenient to choose $\varphi_1 \equiv a\phi$ and $\varphi_2 \equiv a\chi$ as canonical field variables, with corresponding canonical momenta $\Pi_1 = a\dot{\phi}$ and $\Pi_2=a\dot{\chi}$ where a dot represents a derivative with respect to conformal time $\tau$.
The corresponding canonical momenta are
\begin{align}
  \Pi_i &= a\dot{\phi}_i = \dot{\varphi}_i - \frac{\dot{a}}{a}\varphi_i \\
  \Pi_a &= -6M_P^2\dot{a} - \frac{1}{a}\sum_i\Pi_i\varphi_i
\end{align}
In terms of these variables, the Hamiltonian is
\begin{equation}
  \mathcal{H} = \sum_i\frac{\Pi_i^2}{2} + \frac{\lambda}{4}\left(\varphi_1^4 + \frac{2g^2}{\lambda}\varphi_1^2\varphi_2^2 \right) - \frac{1}{12 M_P^2}\left(\Pi_a + \sum_i\frac{\phi_i}{a}\Pi_i\right)^2
\end{equation}
with the constraint
\begin{equation}
  \mathcal{H} = 0 \, .
\end{equation}
This yields the following equations of motion
\begin{align}\label{eqn:eom_scale}
  \frac{\partial\varphi_i}{\partial\tau} &= \Pi_i - \frac{\varphi_i}{6M_P^2a}\left(\Pi_a + \frac{\varphi_i}{a}\right) \\
  \frac{\partial\Pi_i}{\partial\tau}     &= -\frac{\partial\tilde{V}}{\partial\varphi} +\frac{1}{6M_P^2 a}\left(\Pi_a + \frac{\varphi_i}{a}\right)\Pi_i - \frac{\partial \tilde{V}}{\partial\varphi} \\
  \frac{\partial a}{\partial\tau}        &= -\frac{1}{6M_P^2}\left(\Pi_a + \frac{\varphi_i}{a}\Pi_i\right) \\
  \frac{\partial\Pi_a}{\partial\tau}     &= -\frac{\varphi_i}{6M_P^2 a^2}\left(\Pi_a + \frac{\varphi_i}{a}\Pi_i\right)
\end{align}
where $\tilde{V}(\vec{\varphi}) = a^{-4}V(\vec{\phi}) = \frac{\lambda}{4}\left(\varphi_1^4 + \frac{g^2}{2\lambda}\varphi_1^2\varphi_2^2 \right)$.
Note that the canonical momenta for the unscaled field coordinates (see below) $\pi_a = \Pi_a + \frac{\varphi_i}{a} = -6M_P^2\dot{a}$ appears as a damping term in each of the equations.
Alternatively, in unscaled field variables $\phi,\chi$ we have
\begin{align}
  \pi_i &= a^2\dot{\phi}_i\\
  \pi_a &= -6M_P^2\dot{a} 
\end{align}
and
\begin{equation}
  \mathcal{H} = \sum_i\frac{\pi_i^2}{2a^2} + a^4V(\vec{\phi}) - \frac{\Pi_a^2}{12M_P^2}
\end{equation}
with corresponding equations
\begin{align}\label{eqn:eom_noscale}
  \frac{\partial\phi_i}{\partial\tau} &= \frac{\pi_i}{a^2} \\
  \frac{\partial\pi_i}{\partial\tau} &= -a^4 \frac{\partial_iV(\phi)}{\partial\phi_i} \\
  \frac{\partial a}{\partial\tau}     &= -\frac{\Pi_a}{6M_P^2} \\
  \frac{\partial\Pi_a}{\partial\tau}  &= \sum_i\frac{\pi_i^2}{a^3} - a^3V \, .
\end{align}

{\bf Give a better description of initial conditions}
To connect to our lattice simulations, we consider the evolution of a collection of decoupled ballistic trajectories.
These trajectories are centered on the mean values of the fundamental fields in the corresponding lattice simulations, and each one receives a weight corresponding to the probability that the particular value would be realised amongst the subhorizon fluctuations.
The mean value is drawn from the distribution of superhorizon values $\chi_i$, and each ballistic trajectory is then meant to simulate the evolution of the fields on subhorizon scales.
Therefore, we can completely specify the relevant initial data .
For Gaussian subhorizon fluctuations relevant to the case of cosmological preheating, the subhorizon joint PDF can be obtained by the covariance matrix.
To emulate the lattice simulations, we then compute equivalent lattice averages of functions of the basic field variables by evaluating the expectation value over the initial field bundle
\begin{equation}
  \langle \mathcal{O} \rangle(\tau) = \sum_{i=1}^{n_{\rm traj}} P_i\mathcal{O}(\qc(\tau|\qc_{0,i}))
\end{equation}
For example, the extract $\zeta$, we compute $\langle a \rangle$ and $\langle H \rangle$.
We then obtain $\zeta$ by moving to the desired $\langle H \rangle$ slice and extracting $\ln\langle a \rangle$.
In the classical limit, the model we consider is conformally invariant, so that $a^2H = \Pi_a$ is constant when averaged over an oscillation of the fields at the bottom of the potential.
Since we wish to extract the curvature perturbation on fixed $H$ slices, it is therefore convenient to instead consider the smoothed value of the time averaged $\Pi_a$ in order to construct $\zeta$.
This approach should be valid in cases where the most unstable field mode is initially the zero-mode.
Although this is not completely generic, we will argue below that this occurs for a wide range of shapes near the local minima of the scalar field potential.

The computational gain from this approach is tremendous.
Not only can we evolve each of our effective lattice sites independently, allowing for trivial parallelisation, we only need to consider each value of the initial conditions once and then just adjust the initial probability.
In comparison, in a single lattice simulation, we must evolve a sufficient number of points that the initial PDF for the subhorizon degrees of freedom is properly sampled.
Therefore, even without scanning over the mean values within the cube, we must already evolve many copies of trajectories with the same initial conditions (or more precisely with a much finer spacing in initial field values sampling the PDF).
When the ballistic approximation holds, this amounts to solving for a huge amount of redundant information.

We solve the equations of motion~\eqref{eqn:eom_noscale} using a tenth order accurate Gauss-Legendre integrator~\cite{Butcher,Braden}.
This integration scheme is symplectic, preserving quadratic invariants of the action as well as the canonical symplectic 2-form for the Hamiltonian system.
As will be seen, this dynamical system exhibits chaotic behaviour for a range of coupling parameter $2g^2/\lambda$.
Therefore, the preservation of the phase-space structure by our integration scheme is of prime importance when attempting to compute averaged quantities over a cloud of initial trajectories,
since the chaotic motion would quickly destroy preservation of probabilities for an inferior integration scheme.
Due to the extreme accuracy of our integrator, we are able to preserve the Hamiltonian constraint $\mathcal{H}=0$ to machine precision along each individual ballistic trajectory.
\begin{figure}
  \caption{{\bf Do we want this} \emph{Left:} Evolution of the Hamiltonian constraint along each individual trajectory.  The bit flipping and random walk of the errors associated with machine precision are evident. \emph{Right:} Difference in trajectories if we halve the time step, again demonstrating the machine precision level of convergence.}
\end{figure}

%In this sense, it is more natural to consider the equations for the longitudinal and transverse parts of the field {\bf Work out this and do the calculation}.
%An approximate, but less technically involved approach, is to instead consider the radial and angular parts of the field defined through
%\begin{align}
%  \phi &= \sigma\cos\theta \\
%  \chi &= \sigma\sin\theta
%\end{align}

Since our primary goal is to compute the comoving curvature perturbation $\zeta$, in~\figref{fig:pia_waves} we show the evolution of $\Pi_a=a^2H$ as a function of initial superhorizon $\chi_i$ for both our ballistic simulations and full lattice simulations.
\begin{figure}\label{fig:pia_waves}
  \caption{Evolution of the canonical momentum associated with the scale factor $\Pi_a = a^2H$ as a function of conformal time $\tau$ and initial $\chi$ value.  On the left we show the result for individual (unsmoothed) ballistic trajectories, in the middle panel with a smoothing by a Gaussian kernel of width {\bf fill in} and in the bottom panel for full lattice simulations.  For comparison, we also show the computed curvature perturbation $\zeta$ for the ballistic and lattice approaches.}
\end{figure}
\figref{fig:pia_waves} provides a convincing demonstration that the production of density perturbations from preheating modulated by a superhorizon isocurvature mode can (at least in the case of the model described by~\eqref{eqn:}) be understood in terms of ballistic motion of scalar fields oscillating in a potential.
We now explore the creation of this structure in more detail, and demonstrate how the field dynamics and formation of caustics leads to the spiky structure for $\zeta$.
Since caustic formation is an ubiquitous feature of chaotic billiard dynamics, we hope to convince the reader that this mechanism of density perturbation production from preheating will be extremely common.

{\bf Point out the periodicity in the plot.  Will focus on one period in what follows.  Obtaining the basic structural block requires numerics.  We should also be able to get the scaling of the peaks widths, etc through some reasonably simple argument.  The existence of two distinct peaks occurs because there are 2 arms? (could easily check with another model with 3 arms say).  Substructure comes from usual period doubling and extreme coiling of the string as time evolves.}
The periodic structure in $\ln\chi_i$ is evident in~\figref{pia_waves}, so for simplicity we focus on the dynamics of a single period of this structure in what follows.
The existence of the periodicity is easily understood from the linear stability analysis for small fluctuations in $\chi$ around the initial trajectory with $\chi_i=0$, which oscillates solely in the $\phi$ direction.
As is well known, this problem can be reduced to the solution of the Lame equation~\cite{KofmanGreene} up to small corrections associated with the evolution of the scale factor $a$.
Standard techniques from Floquet theory then allow for the stability of various wavenumbers of the $\chi$ perturbations to be analysed.
On finds that for certain ranges of $2g^2/\lambda$ the zero-mode is exponentially unstable, with the $\chi$ zero-mode evolving as $\chi = P(\tau)e^{\mu \tau/T}$ where $P(\tau+T)=P(T)$, $T$ is the period of the $phi$ oscillation, and $\mu$ is known as the Floquet exponent.
From the ballistic viewpoint, this stage is more naturally viewed as a rotation of trajectory's oscillation ``plane'' {\bf line?} in both the $(\phi,\chi)$ and $(\dot{\phi},\dot{\chi})$ planes.
{\bf How do we estimate or understand some of the structure of each individual period in $\ln\chi$?}

{\bf If we time average, the travelling caustics will disappear.  These are probably associated with the things like turning points of the field in the potential or something, so it's reasonable to time-average them.  The frozen in ones won't disappear when we time average.}
\begin{figure}
  \includegraphics[width=0.33\linewidth]{{{lna_tevolve}}}
  \includegraphics[width=0.33\linewidth]{{{hubble_tevolve}}}
  \caption{Evolution of $\ln a$ (\emph{left}) and $H$ (\emph{right}) for a variety of time slices.  As the colors change from light orange to red, the $\tau$ slice the trajectories are extracted on increases.  The creation of coherent structures in both variables relevant for $\zeta$ extraction.}
\end{figure}

\begin{figure}
  \includegraphics[width=0.33\linewidth]{{{lna_caustic}}}
  \caption{Development of caustics in the gravitational plane.  Notice how the caustic locations are associated with the development of the coherent peaks in $\ln a$ and $H$. {\bf Understand why the first caustic to develop has smaller response in $\ln a$ than the second one.  Why doesn't the one that travels along the string produce an effect (ie. what's it associated with)}.}
\end{figure}


\begin{figure}
  \includegraphics[width=3.375in]{{{string-length}}}
  \caption{Length of the phase string projected as measured along each of the individual phase space directions associated with the fields.}  %Here we have defined our line-element on phase space as $ds^2 = d\varphi_1^2+d\varphi_2^2 + d\Pi_1^2 + d\Pi_2^2$.}
\end{figure}

\begin{figure}
  \caption{Response of the phase space variables to small initial perturbations in the modulating field value $\delta\chi$.}
\end{figure}

\begin{figure}
  \includegraphics[width=0.49\linewidth]{{{phase-string-3d_t600}}}
  \includegraphics[width=0.49\linewidth]{{{phase-string-3d_t625}}}
  \caption{Three-Dimensional Projection of the initial phase string into the $(a\phi,a\chi,a\dot{\phi})$ cube.  The full phase string is shown in black.  The projections onto three remaining two-dimensional planes are shown in blue.  Also included for reference are isocontours on constant potential energy, energy in $\phi$ (with $\chi=0$) and ..., along with the isocontour corresponding to the initial value of the energy.  Since the momenta associated with the scale factor $a$ is conserved in time average, these initial isocontours can be viewed as roughly showing the projection of the constraint surface $\mathcal{H}=0$ into the appropriate planes.  From left to right we show the distribution at $\tau = $,$\tau = $, and $\tau = $.}
\end{figure}

\begin{figure}
  \caption{Entropy of projected PDF as a function of time?}
\end{figure}

\begin{figure}
  \caption{Distribution of $\ln a$ for a Gaussian bundle of trajectories centered around a spike (\emph{left}) and without a spike (\emph{right}).}
\end{figure}

{\bf To Do:  Fix the initial Hubble to be the same at the start of each inidividual trajectory?  Might remove one of the ugly splitting issues.}

\subsection{Dynamics in field $\phi,\chi$ plane}

\subsection{Dynamics in gravitational $a,\Pi_a$ plane}
