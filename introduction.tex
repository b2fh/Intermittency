\section{Introduction}
\begin{itemize}
\item Basic idea is the splitting of the dynamics into long- and short-wavelengths (the so called long-short or fast-slow split)
\item Previously the short dynamics has been treated by running full lattice simulations with the long-wavelength modes modelled as a constant
\item The long-dynamics is then treated space-time point by spacetime point as ballistics whose dynamics is determined by the subhorizon (lattice) evolution
\item Since the subhorizon dynamics can be modulated by the
\item This is not conceptually clean in Fourier space!!!!
\item Our novelty is that the short dynamics can be treated as decoupled ballistic trajectories as well
\item This provides a novel conceptual framework from which to understand production of density perturbations from preheating
\item Also presents nice parallels with the collapse of cold dark matter (until the discrete nature of the dark matter particles becomes important)
\item Tremendous computational gains relative to the lattice approach.  For a lattice we must run a large number of points so that the sub-horizon PDF is correctly approximated by the lattice sites.  Here, we can run a much smaller number of trajectories and simply attach of probability weight to each of them to get the smoothed lattice dynamics.  
\end{itemize}
