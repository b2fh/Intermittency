\section{Chaotic Structures in Lattice Simulations and the Ballistic Approximation}

\begin{itemize}
\item Show nice figures (ie. the giga-figure) from full lattice simulations
\item Include the equivalent results if we instead use the ballistic approach to show that they agree.
\item In the full lattice description, we must track all of the individual lattice sites simultaneously as a very high-dimensional PDF (as an approximation to doing the PDFunctionals for full fields)
\item The ballistics allows us to reduce the study of this phase space into an extremely reduced phase space where we just need the one-point (joint) PDFs for the variables at each lattice site.  During the ballistic regime, this mapping is rigourous.  In non-ballistic regime will have leakage out of the reduced PDF due to couplings from derivative terms.
\item Evolving the individual trajectories is the Langevin approach (with no noise term after inflation).  Getting the PDFs is the Fokker-Planck approach.
\item Mention that this general framework also applies during inflation and provides a nice unification.
\item Transition in using rest of the paper to explore ballistic dynamics.
\end{itemize}

\begin{figure}
  \caption{Final comoving curvature perturbation as a function of modulating fields associated with the superhorizon isocurvature direction $\chi$ and coupling constant $g^2/\lambda$.}
\end{figure}

\begin{figure}
  \caption{Evolution of $a^2H$ (or $a^4\rho$) as a function of time from lattice simulations (left) and from ballistics (right).}
\end{figure}

\begin{figure}
  \caption{Effective nonGaussianity function after smoothing over an appropriate scale. (could put this in the section with actual maps)}
\end{figure}

\begin{figure}
  \caption{Distribution of mean contributions to the scalar field energy density as a function of mean scale factor, obtained from a full resolved numerical lattice simulation}
\end{figure}

\begin{figure}
  \caption{Evolution of gravitational ``momentum'' along a spike and non-spike trajectory.  {\bf For comparison we also include the results from an individual trajectory and an averaged trajectory bundle in a ballistic simulation.}}
\end{figure}
